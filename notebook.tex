
% Default to the notebook output style

    


% Inherit from the specified cell style.




    
\documentclass[11pt]{article}

    
    
    \usepackage[T1]{fontenc}
    % Nicer default font (+ math font) than Computer Modern for most use cases
    \usepackage{mathpazo}

    % Basic figure setup, for now with no caption control since it's done
    % automatically by Pandoc (which extracts ![](path) syntax from Markdown).
    \usepackage{graphicx}
    % We will generate all images so they have a width \maxwidth. This means
    % that they will get their normal width if they fit onto the page, but
    % are scaled down if they would overflow the margins.
    \makeatletter
    \def\maxwidth{\ifdim\Gin@nat@width>\linewidth\linewidth
    \else\Gin@nat@width\fi}
    \makeatother
    \let\Oldincludegraphics\includegraphics
    % Set max figure width to be 80% of text width, for now hardcoded.
    \renewcommand{\includegraphics}[1]{\Oldincludegraphics[width=.8\maxwidth]{#1}}
    % Ensure that by default, figures have no caption (until we provide a
    % proper Figure object with a Caption API and a way to capture that
    % in the conversion process - todo).
    \usepackage{caption}
    \DeclareCaptionLabelFormat{nolabel}{}
    \captionsetup{labelformat=nolabel}

    \usepackage{adjustbox} % Used to constrain images to a maximum size 
    \usepackage{xcolor} % Allow colors to be defined
    \usepackage{enumerate} % Needed for markdown enumerations to work
    \usepackage{geometry} % Used to adjust the document margins
    \usepackage{amsmath} % Equations
    \usepackage{amssymb} % Equations
    \usepackage{textcomp} % defines textquotesingle
    % Hack from http://tex.stackexchange.com/a/47451/13684:
    \AtBeginDocument{%
        \def\PYZsq{\textquotesingle}% Upright quotes in Pygmentized code
    }
    \usepackage{upquote} % Upright quotes for verbatim code
    \usepackage{eurosym} % defines \euro
    \usepackage[mathletters]{ucs} % Extended unicode (utf-8) support
    \usepackage[utf8x]{inputenc} % Allow utf-8 characters in the tex document
    \usepackage{fancyvrb} % verbatim replacement that allows latex
    \usepackage{grffile} % extends the file name processing of package graphics 
                         % to support a larger range 
    % The hyperref package gives us a pdf with properly built
    % internal navigation ('pdf bookmarks' for the table of contents,
    % internal cross-reference links, web links for URLs, etc.)
    \usepackage{hyperref}
    \usepackage{longtable} % longtable support required by pandoc >1.10
    \usepackage{booktabs}  % table support for pandoc > 1.12.2
    \usepackage[inline]{enumitem} % IRkernel/repr support (it uses the enumerate* environment)
    \usepackage[normalem]{ulem} % ulem is needed to support strikethroughs (\sout)
                                % normalem makes italics be italics, not underlines
    

    
    
    % Colors for the hyperref package
    \definecolor{urlcolor}{rgb}{0,.145,.698}
    \definecolor{linkcolor}{rgb}{.71,0.21,0.01}
    \definecolor{citecolor}{rgb}{.12,.54,.11}

    % ANSI colors
    \definecolor{ansi-black}{HTML}{3E424D}
    \definecolor{ansi-black-intense}{HTML}{282C36}
    \definecolor{ansi-red}{HTML}{E75C58}
    \definecolor{ansi-red-intense}{HTML}{B22B31}
    \definecolor{ansi-green}{HTML}{00A250}
    \definecolor{ansi-green-intense}{HTML}{007427}
    \definecolor{ansi-yellow}{HTML}{DDB62B}
    \definecolor{ansi-yellow-intense}{HTML}{B27D12}
    \definecolor{ansi-blue}{HTML}{208FFB}
    \definecolor{ansi-blue-intense}{HTML}{0065CA}
    \definecolor{ansi-magenta}{HTML}{D160C4}
    \definecolor{ansi-magenta-intense}{HTML}{A03196}
    \definecolor{ansi-cyan}{HTML}{60C6C8}
    \definecolor{ansi-cyan-intense}{HTML}{258F8F}
    \definecolor{ansi-white}{HTML}{C5C1B4}
    \definecolor{ansi-white-intense}{HTML}{A1A6B2}

    % commands and environments needed by pandoc snippets
    % extracted from the output of `pandoc -s`
    \providecommand{\tightlist}{%
      \setlength{\itemsep}{0pt}\setlength{\parskip}{0pt}}
    \DefineVerbatimEnvironment{Highlighting}{Verbatim}{commandchars=\\\{\}}
    % Add ',fontsize=\small' for more characters per line
    \newenvironment{Shaded}{}{}
    \newcommand{\KeywordTok}[1]{\textcolor[rgb]{0.00,0.44,0.13}{\textbf{{#1}}}}
    \newcommand{\DataTypeTok}[1]{\textcolor[rgb]{0.56,0.13,0.00}{{#1}}}
    \newcommand{\DecValTok}[1]{\textcolor[rgb]{0.25,0.63,0.44}{{#1}}}
    \newcommand{\BaseNTok}[1]{\textcolor[rgb]{0.25,0.63,0.44}{{#1}}}
    \newcommand{\FloatTok}[1]{\textcolor[rgb]{0.25,0.63,0.44}{{#1}}}
    \newcommand{\CharTok}[1]{\textcolor[rgb]{0.25,0.44,0.63}{{#1}}}
    \newcommand{\StringTok}[1]{\textcolor[rgb]{0.25,0.44,0.63}{{#1}}}
    \newcommand{\CommentTok}[1]{\textcolor[rgb]{0.38,0.63,0.69}{\textit{{#1}}}}
    \newcommand{\OtherTok}[1]{\textcolor[rgb]{0.00,0.44,0.13}{{#1}}}
    \newcommand{\AlertTok}[1]{\textcolor[rgb]{1.00,0.00,0.00}{\textbf{{#1}}}}
    \newcommand{\FunctionTok}[1]{\textcolor[rgb]{0.02,0.16,0.49}{{#1}}}
    \newcommand{\RegionMarkerTok}[1]{{#1}}
    \newcommand{\ErrorTok}[1]{\textcolor[rgb]{1.00,0.00,0.00}{\textbf{{#1}}}}
    \newcommand{\NormalTok}[1]{{#1}}
    
    % Additional commands for more recent versions of Pandoc
    \newcommand{\ConstantTok}[1]{\textcolor[rgb]{0.53,0.00,0.00}{{#1}}}
    \newcommand{\SpecialCharTok}[1]{\textcolor[rgb]{0.25,0.44,0.63}{{#1}}}
    \newcommand{\VerbatimStringTok}[1]{\textcolor[rgb]{0.25,0.44,0.63}{{#1}}}
    \newcommand{\SpecialStringTok}[1]{\textcolor[rgb]{0.73,0.40,0.53}{{#1}}}
    \newcommand{\ImportTok}[1]{{#1}}
    \newcommand{\DocumentationTok}[1]{\textcolor[rgb]{0.73,0.13,0.13}{\textit{{#1}}}}
    \newcommand{\AnnotationTok}[1]{\textcolor[rgb]{0.38,0.63,0.69}{\textbf{\textit{{#1}}}}}
    \newcommand{\CommentVarTok}[1]{\textcolor[rgb]{0.38,0.63,0.69}{\textbf{\textit{{#1}}}}}
    \newcommand{\VariableTok}[1]{\textcolor[rgb]{0.10,0.09,0.49}{{#1}}}
    \newcommand{\ControlFlowTok}[1]{\textcolor[rgb]{0.00,0.44,0.13}{\textbf{{#1}}}}
    \newcommand{\OperatorTok}[1]{\textcolor[rgb]{0.40,0.40,0.40}{{#1}}}
    \newcommand{\BuiltInTok}[1]{{#1}}
    \newcommand{\ExtensionTok}[1]{{#1}}
    \newcommand{\PreprocessorTok}[1]{\textcolor[rgb]{0.74,0.48,0.00}{{#1}}}
    \newcommand{\AttributeTok}[1]{\textcolor[rgb]{0.49,0.56,0.16}{{#1}}}
    \newcommand{\InformationTok}[1]{\textcolor[rgb]{0.38,0.63,0.69}{\textbf{\textit{{#1}}}}}
    \newcommand{\WarningTok}[1]{\textcolor[rgb]{0.38,0.63,0.69}{\textbf{\textit{{#1}}}}}
    
    
    % Define a nice break command that doesn't care if a line doesn't already
    % exist.
    \def\br{\hspace*{\fill} \\* }
    % Math Jax compatability definitions
    \def\gt{>}
    \def\lt{<}
    % Document parameters
    \title{Relazione}
    
    
    

    % Pygments definitions
    
\makeatletter
\def\PY@reset{\let\PY@it=\relax \let\PY@bf=\relax%
    \let\PY@ul=\relax \let\PY@tc=\relax%
    \let\PY@bc=\relax \let\PY@ff=\relax}
\def\PY@tok#1{\csname PY@tok@#1\endcsname}
\def\PY@toks#1+{\ifx\relax#1\empty\else%
    \PY@tok{#1}\expandafter\PY@toks\fi}
\def\PY@do#1{\PY@bc{\PY@tc{\PY@ul{%
    \PY@it{\PY@bf{\PY@ff{#1}}}}}}}
\def\PY#1#2{\PY@reset\PY@toks#1+\relax+\PY@do{#2}}

\expandafter\def\csname PY@tok@w\endcsname{\def\PY@tc##1{\textcolor[rgb]{0.73,0.73,0.73}{##1}}}
\expandafter\def\csname PY@tok@c\endcsname{\let\PY@it=\textit\def\PY@tc##1{\textcolor[rgb]{0.25,0.50,0.50}{##1}}}
\expandafter\def\csname PY@tok@cp\endcsname{\def\PY@tc##1{\textcolor[rgb]{0.74,0.48,0.00}{##1}}}
\expandafter\def\csname PY@tok@k\endcsname{\let\PY@bf=\textbf\def\PY@tc##1{\textcolor[rgb]{0.00,0.50,0.00}{##1}}}
\expandafter\def\csname PY@tok@kp\endcsname{\def\PY@tc##1{\textcolor[rgb]{0.00,0.50,0.00}{##1}}}
\expandafter\def\csname PY@tok@kt\endcsname{\def\PY@tc##1{\textcolor[rgb]{0.69,0.00,0.25}{##1}}}
\expandafter\def\csname PY@tok@o\endcsname{\def\PY@tc##1{\textcolor[rgb]{0.40,0.40,0.40}{##1}}}
\expandafter\def\csname PY@tok@ow\endcsname{\let\PY@bf=\textbf\def\PY@tc##1{\textcolor[rgb]{0.67,0.13,1.00}{##1}}}
\expandafter\def\csname PY@tok@nb\endcsname{\def\PY@tc##1{\textcolor[rgb]{0.00,0.50,0.00}{##1}}}
\expandafter\def\csname PY@tok@nf\endcsname{\def\PY@tc##1{\textcolor[rgb]{0.00,0.00,1.00}{##1}}}
\expandafter\def\csname PY@tok@nc\endcsname{\let\PY@bf=\textbf\def\PY@tc##1{\textcolor[rgb]{0.00,0.00,1.00}{##1}}}
\expandafter\def\csname PY@tok@nn\endcsname{\let\PY@bf=\textbf\def\PY@tc##1{\textcolor[rgb]{0.00,0.00,1.00}{##1}}}
\expandafter\def\csname PY@tok@ne\endcsname{\let\PY@bf=\textbf\def\PY@tc##1{\textcolor[rgb]{0.82,0.25,0.23}{##1}}}
\expandafter\def\csname PY@tok@nv\endcsname{\def\PY@tc##1{\textcolor[rgb]{0.10,0.09,0.49}{##1}}}
\expandafter\def\csname PY@tok@no\endcsname{\def\PY@tc##1{\textcolor[rgb]{0.53,0.00,0.00}{##1}}}
\expandafter\def\csname PY@tok@nl\endcsname{\def\PY@tc##1{\textcolor[rgb]{0.63,0.63,0.00}{##1}}}
\expandafter\def\csname PY@tok@ni\endcsname{\let\PY@bf=\textbf\def\PY@tc##1{\textcolor[rgb]{0.60,0.60,0.60}{##1}}}
\expandafter\def\csname PY@tok@na\endcsname{\def\PY@tc##1{\textcolor[rgb]{0.49,0.56,0.16}{##1}}}
\expandafter\def\csname PY@tok@nt\endcsname{\let\PY@bf=\textbf\def\PY@tc##1{\textcolor[rgb]{0.00,0.50,0.00}{##1}}}
\expandafter\def\csname PY@tok@nd\endcsname{\def\PY@tc##1{\textcolor[rgb]{0.67,0.13,1.00}{##1}}}
\expandafter\def\csname PY@tok@s\endcsname{\def\PY@tc##1{\textcolor[rgb]{0.73,0.13,0.13}{##1}}}
\expandafter\def\csname PY@tok@sd\endcsname{\let\PY@it=\textit\def\PY@tc##1{\textcolor[rgb]{0.73,0.13,0.13}{##1}}}
\expandafter\def\csname PY@tok@si\endcsname{\let\PY@bf=\textbf\def\PY@tc##1{\textcolor[rgb]{0.73,0.40,0.53}{##1}}}
\expandafter\def\csname PY@tok@se\endcsname{\let\PY@bf=\textbf\def\PY@tc##1{\textcolor[rgb]{0.73,0.40,0.13}{##1}}}
\expandafter\def\csname PY@tok@sr\endcsname{\def\PY@tc##1{\textcolor[rgb]{0.73,0.40,0.53}{##1}}}
\expandafter\def\csname PY@tok@ss\endcsname{\def\PY@tc##1{\textcolor[rgb]{0.10,0.09,0.49}{##1}}}
\expandafter\def\csname PY@tok@sx\endcsname{\def\PY@tc##1{\textcolor[rgb]{0.00,0.50,0.00}{##1}}}
\expandafter\def\csname PY@tok@m\endcsname{\def\PY@tc##1{\textcolor[rgb]{0.40,0.40,0.40}{##1}}}
\expandafter\def\csname PY@tok@gh\endcsname{\let\PY@bf=\textbf\def\PY@tc##1{\textcolor[rgb]{0.00,0.00,0.50}{##1}}}
\expandafter\def\csname PY@tok@gu\endcsname{\let\PY@bf=\textbf\def\PY@tc##1{\textcolor[rgb]{0.50,0.00,0.50}{##1}}}
\expandafter\def\csname PY@tok@gd\endcsname{\def\PY@tc##1{\textcolor[rgb]{0.63,0.00,0.00}{##1}}}
\expandafter\def\csname PY@tok@gi\endcsname{\def\PY@tc##1{\textcolor[rgb]{0.00,0.63,0.00}{##1}}}
\expandafter\def\csname PY@tok@gr\endcsname{\def\PY@tc##1{\textcolor[rgb]{1.00,0.00,0.00}{##1}}}
\expandafter\def\csname PY@tok@ge\endcsname{\let\PY@it=\textit}
\expandafter\def\csname PY@tok@gs\endcsname{\let\PY@bf=\textbf}
\expandafter\def\csname PY@tok@gp\endcsname{\let\PY@bf=\textbf\def\PY@tc##1{\textcolor[rgb]{0.00,0.00,0.50}{##1}}}
\expandafter\def\csname PY@tok@go\endcsname{\def\PY@tc##1{\textcolor[rgb]{0.53,0.53,0.53}{##1}}}
\expandafter\def\csname PY@tok@gt\endcsname{\def\PY@tc##1{\textcolor[rgb]{0.00,0.27,0.87}{##1}}}
\expandafter\def\csname PY@tok@err\endcsname{\def\PY@bc##1{\setlength{\fboxsep}{0pt}\fcolorbox[rgb]{1.00,0.00,0.00}{1,1,1}{\strut ##1}}}
\expandafter\def\csname PY@tok@kc\endcsname{\let\PY@bf=\textbf\def\PY@tc##1{\textcolor[rgb]{0.00,0.50,0.00}{##1}}}
\expandafter\def\csname PY@tok@kd\endcsname{\let\PY@bf=\textbf\def\PY@tc##1{\textcolor[rgb]{0.00,0.50,0.00}{##1}}}
\expandafter\def\csname PY@tok@kn\endcsname{\let\PY@bf=\textbf\def\PY@tc##1{\textcolor[rgb]{0.00,0.50,0.00}{##1}}}
\expandafter\def\csname PY@tok@kr\endcsname{\let\PY@bf=\textbf\def\PY@tc##1{\textcolor[rgb]{0.00,0.50,0.00}{##1}}}
\expandafter\def\csname PY@tok@bp\endcsname{\def\PY@tc##1{\textcolor[rgb]{0.00,0.50,0.00}{##1}}}
\expandafter\def\csname PY@tok@fm\endcsname{\def\PY@tc##1{\textcolor[rgb]{0.00,0.00,1.00}{##1}}}
\expandafter\def\csname PY@tok@vc\endcsname{\def\PY@tc##1{\textcolor[rgb]{0.10,0.09,0.49}{##1}}}
\expandafter\def\csname PY@tok@vg\endcsname{\def\PY@tc##1{\textcolor[rgb]{0.10,0.09,0.49}{##1}}}
\expandafter\def\csname PY@tok@vi\endcsname{\def\PY@tc##1{\textcolor[rgb]{0.10,0.09,0.49}{##1}}}
\expandafter\def\csname PY@tok@vm\endcsname{\def\PY@tc##1{\textcolor[rgb]{0.10,0.09,0.49}{##1}}}
\expandafter\def\csname PY@tok@sa\endcsname{\def\PY@tc##1{\textcolor[rgb]{0.73,0.13,0.13}{##1}}}
\expandafter\def\csname PY@tok@sb\endcsname{\def\PY@tc##1{\textcolor[rgb]{0.73,0.13,0.13}{##1}}}
\expandafter\def\csname PY@tok@sc\endcsname{\def\PY@tc##1{\textcolor[rgb]{0.73,0.13,0.13}{##1}}}
\expandafter\def\csname PY@tok@dl\endcsname{\def\PY@tc##1{\textcolor[rgb]{0.73,0.13,0.13}{##1}}}
\expandafter\def\csname PY@tok@s2\endcsname{\def\PY@tc##1{\textcolor[rgb]{0.73,0.13,0.13}{##1}}}
\expandafter\def\csname PY@tok@sh\endcsname{\def\PY@tc##1{\textcolor[rgb]{0.73,0.13,0.13}{##1}}}
\expandafter\def\csname PY@tok@s1\endcsname{\def\PY@tc##1{\textcolor[rgb]{0.73,0.13,0.13}{##1}}}
\expandafter\def\csname PY@tok@mb\endcsname{\def\PY@tc##1{\textcolor[rgb]{0.40,0.40,0.40}{##1}}}
\expandafter\def\csname PY@tok@mf\endcsname{\def\PY@tc##1{\textcolor[rgb]{0.40,0.40,0.40}{##1}}}
\expandafter\def\csname PY@tok@mh\endcsname{\def\PY@tc##1{\textcolor[rgb]{0.40,0.40,0.40}{##1}}}
\expandafter\def\csname PY@tok@mi\endcsname{\def\PY@tc##1{\textcolor[rgb]{0.40,0.40,0.40}{##1}}}
\expandafter\def\csname PY@tok@il\endcsname{\def\PY@tc##1{\textcolor[rgb]{0.40,0.40,0.40}{##1}}}
\expandafter\def\csname PY@tok@mo\endcsname{\def\PY@tc##1{\textcolor[rgb]{0.40,0.40,0.40}{##1}}}
\expandafter\def\csname PY@tok@ch\endcsname{\let\PY@it=\textit\def\PY@tc##1{\textcolor[rgb]{0.25,0.50,0.50}{##1}}}
\expandafter\def\csname PY@tok@cm\endcsname{\let\PY@it=\textit\def\PY@tc##1{\textcolor[rgb]{0.25,0.50,0.50}{##1}}}
\expandafter\def\csname PY@tok@cpf\endcsname{\let\PY@it=\textit\def\PY@tc##1{\textcolor[rgb]{0.25,0.50,0.50}{##1}}}
\expandafter\def\csname PY@tok@c1\endcsname{\let\PY@it=\textit\def\PY@tc##1{\textcolor[rgb]{0.25,0.50,0.50}{##1}}}
\expandafter\def\csname PY@tok@cs\endcsname{\let\PY@it=\textit\def\PY@tc##1{\textcolor[rgb]{0.25,0.50,0.50}{##1}}}

\def\PYZbs{\char`\\}
\def\PYZus{\char`\_}
\def\PYZob{\char`\{}
\def\PYZcb{\char`\}}
\def\PYZca{\char`\^}
\def\PYZam{\char`\&}
\def\PYZlt{\char`\<}
\def\PYZgt{\char`\>}
\def\PYZsh{\char`\#}
\def\PYZpc{\char`\%}
\def\PYZdl{\char`\$}
\def\PYZhy{\char`\-}
\def\PYZsq{\char`\'}
\def\PYZdq{\char`\"}
\def\PYZti{\char`\~}
% for compatibility with earlier versions
\def\PYZat{@}
\def\PYZlb{[}
\def\PYZrb{]}
\makeatother


    % Exact colors from NB
    \definecolor{incolor}{rgb}{0.0, 0.0, 0.5}
    \definecolor{outcolor}{rgb}{0.545, 0.0, 0.0}



    
    % Prevent overflowing lines due to hard-to-break entities
    \sloppy 
    % Setup hyperref package
    \hypersetup{
      breaklinks=true,  % so long urls are correctly broken across lines
      colorlinks=true,
      urlcolor=urlcolor,
      linkcolor=linkcolor,
      citecolor=citecolor,
      }
    % Slightly bigger margins than the latex defaults
    
    \geometry{verbose,tmargin=1in,bmargin=1in,lmargin=1in,rmargin=1in}
    
    

    \begin{document}
    
    
    \maketitle
    
    

    
    Table of Contents{}

{{1~~}Progetto Google-Revenue Kaggle}

{{2~~}Glossario}

{{3~~}Asserzioni sui dati}

{{4~~}Data Load and Parsing}

{{4.1~~}Introduzione}

{{4.2~~}Unpacking e Unfolding}

{{4.3~~}Conclusioni}

{{5~~}Preprocessing}

{{5.1~~}Introduzione}

{{5.2~~}Funzione encode\_cats}

{{5.3~~}Funzione group\_me}

{{5.4~~}Conclusioni}

{{6~~}Training e Scoring}

{{6.1~~}Introduzione}

{{6.2~~}Lin-reg}

{{6.3~~}LightGBM}

{{6.4~~}Conclusioni}

{{7~~}Scelte}

{{7.1~~}(No) EDA}

{{7.2~~}transactionRevenue e totalTransactionRevenue}

{{7.3~~}Inefficienza della moda}

{{7.4~~}Rischio di overfit con productSKU}

{{7.5~~}Molte feature sintetiche vs poche features reali}

{{7.6~~}Cross-validation}

{{8~~}Features aggiuntive}

{{9~~}Features future}

{{10~~}I numeri di questa competizione}

{{11~~}Considerazioni}

{{12~~}Dati raccolti}

{{13~~}Esperimenti}

{{13.1~~}Esperimento 1:}

{{13.1.1~~}Aumentare le foglie in una random forest migliora lo score in
virtù della teoria che ci sta dietro}

{{13.1.2~~}Conclusione inaspettata}

{{13.2~~}Esperimento 2}

{{13.2.1~~}su piccoli dataset la LR performa meglio di LightGBM(senza
transactionRevenue)}

{{13.2.2~~}Conclusione}

{{13.3~~}Esperimento 3}

{{13.3.1~~}su piccoli dataset LightGBM con singolo albero non ha
eguali(con transactionRevenue)}

{{13.3.2~~}Conclusione}

{{13.4~~}Esperimento 4}

{{13.4.1~~}miglior n\_leaves per LGBM}

{{13.4.2~~}Conclusione}

{{13.5~~}Esperimento 5}

{{13.5.1~~}la moda per le cats è più potente(e sensata) della media}

{{13.5.2~~}Conclusioni}

{{13.6~~}Esperimento 6}

{{13.6.1~~}Aumentare le foglie di LGBM o LGBM\_rf non porta
miglioramento dopo un certo treshold}

{{13.6.2~~}Conclusioni}

    \section{Progetto Google-Revenue
Kaggle}\label{progetto-google-revenue-kaggle}

\begin{center}\rule{0.5\linewidth}{\linethickness}\end{center}

corso: web-intelligence

autore: bernardi riccardo

matricola: 864018

    \section{Glossario}\label{glossario}

\begin{center}\rule{0.5\linewidth}{\linethickness}\end{center}

\begin{itemize}
\tightlist
\item
  lightgbm = in alcuni casi indica l' albero di regressione, è esplicito
  rispetto al contesto
\item
  features = colonne o dimensioni del dataset
\item
  transactionRevenue = target deprecato, ora predittore
\item
  totalTransactionRevenue = target odierno
\item
  sampling = mescolamento
\item
  training = addestramento di un regressore
\item
  EDA = analisi esplorativa dei dati, prima analisi
\item
  regressore = predittore
\item
  LB = leaderboard di kaggle per questa competizione
\item
  RMSE = errore nella predizione/validazione
\item
  score = minimo/basso RMSE, minimo errore, posizione alta in LB
\end{itemize}

    \section{Asserzioni sui dati}\label{asserzioni-sui-dati}

\begin{center}\rule{0.5\linewidth}{\linethickness}\end{center}

Una volta osservati i dati si può notare che ci sono molte colonne che
non sono utilizzabili, queste sono quelle che sono presenti solo in uno
dei due set, in tal caso non servono perchè un modello trainato su
quelle features non verrà sfruttato nel test visto che non sono presenti
lì e vice-versa non saranno utili perchè non possono essere trainate.
Altre colonne non usabili sono quelle diverse dall' ID che però sono
univoche, non aggiungono informazioni poichè sono aleatorie e non vi si
può dedurre un pattern. Le colonne che hanno un solo valore inoltre sono
ugualmente da scartare poichè hanno una colonna costante che non
aggiunge informazione poichè tutte le righe lo possiedono. Si vuole poi
fare una importante osservazione sugli utenti e cioè che un utente
avente più entries non sempre abbia un comportamento coerente. Si
risolverà il comportamento incoerente degli utenti mediando i valori
numerici poichè diminuisco la varianza e poi si prenderà la moda dei
valori categoriali. Questo perchè fare la media di dati categoriali non
permetterebbe poi di avere una funzione inversa che mappa un reale su
dei numeri interi. Ha senso prendere l' elemento più ripetuto poichè in
tal caso con un semplice dict poi si può mappare dall' intero alla
categoria. Si assumerà perciò che un utente possa avere a volte uno
smartphone a volte un altro(per esempio). Per migliorare quindi, dopo
queste assunzioni, lo score si è deciso di droppare le colonne costanti,
con valori unici e le colonne che non siano presenti in entrambi i
datasets a parte la colonna del target. I dati non presenti vengono
assunti a 0.0 poichè è coerente e consistente che sia così, i customers
che hanno nan in una colonna significa che un certo campo numerico ha
valore 0, in un campo categoriale invece mettiamo 0 per comodità e poi
verrà valutato attraverso la moda quindi se è un valore outlier verrà
mascherato dalla robustezza di questo funzionale statistico.

    \section{Data Load and Parsing}\label{data-load-and-parsing}

\begin{center}\rule{0.5\linewidth}{\linethickness}\end{center}

\subsection{Introduzione}\label{introduzione}

La parte di data loading ma soprattutto parsing dopo l' aggiornamento
del dataset, avvenuto per un data leakage, è diventata una parte core di
questa analisi poichè l' encoding delle informazioni è diventato sempre
più difficoltoso a causa di una maggiore densità e innestatezza dei
dati. Si è iniziato osservando che il training set conteneva json da
normalizzare e parsare come nel precedente dataset ma in aggiunta erano
comparse due nuove mastodontiche colonne che comprendevano liste di
jsons(in realtà liste di dicts che potevano contenere altre liste e
altri dicts, ma ad uno sguardo iniziale si poteva incappare nell' errore
di vederli appunto come jsons). Il solo unfolding di queste ultime 2
colonne ha richiesto molta struttura e codice( \(\tilde{}\) 200 righe,
in python!) poichè non era assolutamente pensabile nè di droppare tutte
le colonne "dense" nè di unfoldarle a mano, si è perciò optato per
costruire delle funzioni che facessero auto-discovering di liste o dicts
all interno delle colonne ma in maniera ricorsiva poichè una volta
aperta una colonna e aggiunta a destra bisognava ri-discoverare nell'
intero dataset se era possibile continuare ad unfoldare liste e/o
dicts(in realtà non è implementato ricorsivamente a causa del fatto che
python non regge bene questo paradigma, vedremo che vengono usati cicli
while e vengono posti vincoli sull' unfolding). Una volta costruite le
funzioni di auto-discovering si è però notato che mancava qualcosa:
alcune colonne venivano riconosciute come lists e/o dicts ma altre no...
Il problema è stato fixato usando 2 try...catch... innestati che
discoverano le colonne controllando il tipo con eval() oppure senza
poichè si è notato che una volta unfoldato un dict viene rilasciato il
suo valore come stringa e non come dict(nel caso di dicts innestati con
anche liste innestate) e poi il problema si ripropone con liste
innestate. Il problema non era risolvibile chiedendo il semplice dtypes
poichè tutte le colonne più complesse di un int64 in numpy sono
catalogate come object... per quanto invece riguarda chiederne il type
non è sufficiente perchè appunto se è una lista encodata come stringa
non verrà unfoldata quindi qui si capisce il senso del try...catch...
cioè so che se chiedo l' eval di una lista otterrò un errore e
consapevolmente lo catturo poi se è una lista/dict la unfolderà, se
invece è altro passerà oltre.

\subsection{Unpacking e Unfolding}\label{unpacking-e-unfolding}

A questo punto si vuole introdurre la differenza tra unpacking e
unfolding(nell' introduzione per semplicità non è stata trattata): nel
caso di una colonna che con l auto-discovering si è scoperto contenere
una lista di dicts allora si preferirà avere un solo dicts che come
value contiene un array di tutti i valori che prima erano contenuti in
singoli dicts con un value cioè :
\(dict[key->value]_1,...,dict[key->value]_n -> dict[key->value_1,...,value_n] \forall\text{ } key\text{ } \in dict \in list\).
Il senso di fare ciò è poi avere una interrrogazione molto molto più
semplice poichè per esempio se si vuole estrarre la media del costo dei
singolioggetti comprati da un utente in una sessione basterà fare
\(mean[dict[cost]]\) piuttosto che dover ciclare e cercare gli elementi
con il rischio di vedersi throwate eccezioni se un dict solo non
contiene l' elemento ricercato. Quindi l' unpacking è proprio questo, si
ha la traduzione da una lista di dizionari ad un solo dizionario. L
unfolding invece è un operazione simile poichè si tratta sempre della
trasformazione con un aumento dell' espressività dei dati, in
particolare la unfold non tratta una sola colonna come la unpack ma
tratta un set di colonne che contengono solo dicts e le trasforma in
opportunamente usando una ri-implementazione del costrutto funzionale
map() per normalizzarle, trasformarle in un dataframe (ogni colonna) e
poi aggiungerle al dataframe di partenza avendo cura di non lasciare
dati obsoleti indietro(droppando le colonne vecchie e lasciando le
nuove), tutto ciò avviene in maniera molto molto efficiente poichè si
usano sempre colonne di dataframe che vengono modificate in maniera
parallela e non si passa mai nulla per copia e non si scende mai di
livello presso strutture dati a prestazioni peggiori, in questa maniera
la modifica e il merge delle colonne avviene in tempo praticamente
costante perchè si sfrutta il passaggio per reference delle colonne di
un dataframe, il risultato è che una volta caricato dall' HD il
dataframe poi si ha un ambiente che permette tante operazioni in tempo
quasi nullo.

\subsection{Conclusioni}\label{conclusioni}

Vengono aggiunti vari controlli , caricate le colonne che sono
effettivamente jsons e inoltre vengono droppate colonne uniche e colonne
costanti, il tutto per dare all utente finale una sola ed unica funzione
che permette di essere chiamata con pochi parametri e fa già di per sè
un lavoro mastodontico di prima pulizia dei dati ma con eleganza perchè
lo fa in maniera trasparente.

    \section{Preprocessing}\label{preprocessing}

\begin{center}\rule{0.5\linewidth}{\linethickness}\end{center}

\subsection{Introduzione}\label{introduzione}

Dopo una prima pulizia dei dati viene la parte più importante cioè la
strategia per rendere i dati il più possibile espressivi attraverso
opportune codifiche. Sceglieremo, non essendoci dati ordinali, di
encodare tramite one-hot-encoding i dati categoriali. Questa operazione
è molto costosa in termini computazionali e spaziali poichè ogni
inefficienza ha un costo esponenziale, per esempio inizialmente si
facevano più dataframes che poi venivano ri-arrangiati droppando colonne
fino al raggiungimento delle colonne ricercate mentre ora si punta a
fare un merge tra il dataframe iniziale, il nuovo dataframe dei dati
categoriali categoriali encodato e poi droppando le colonne encodate,
questa operazione avviene ovviamente in tempo costante perchè è solo il
costo di mettere un poiter dall' ultima colonna di un dataframe che
punta alla prima colonna(leggasi indirizzo di memoria in 0) del secondo
dataframe. Si fa notare quindi l' utilizzo efficiente dei dataframes. Il
vantaggio infatti dei dataframe pandas è la possibilità di essere
interrogati con colonne, indici, array di colonne, interrogazioni
simil-sql e la grande efficienza con la quale si possono unire più
dataframes orizzontalmente o verticalmente in tempo praticamente
costante poichè in realtà si lavora a basso livello con pointers. Si fa
notare inoltre l' utilizzo di pd.Series invece che di list poichè dopo
attenti tests si è notato essere molto più efficienti. L' utilizzo poi
delle matrici numpy è motivato dall' efficienza dei calcoli e la
semplicità con la quale si può passare una matrice ad un dataframe,
questo passaggio è piccolo ma lo si può notare direttamente nell'
encoding delle colonne, è una parte minuscola ma cruciale e averla
implementata così è stato importantissimo per ottenere efficienza. Una
volta finito l' encoding si passa ad un altra fase anch' essa molto
importante cioè vengono uniti gli utenti che hanno più istanze. Si
decide di trasformare il dataframe iniziale in 2 diversi dataframes in
cui si fa rispettivamente la media dei valori numerici e la moda dei
valori categoriali encodati. In tutto ciò, che potrebbe sembrare un
operazione trasparente in realtà(ed effettivamente dal lato utente lo è)
ci sono diversi problemi di corruzione dei dati dovuti anche al fatto
che viene modificato l' indice dei due dataframes. Perdendo l' indice
originale anticipo già che si avrebbero problemi nelle successive parti
del codice che lavorano appunto con l' indice numerico predefinito
perciò si è preparata una procedura generica per resettarlo in maniera
efficace. A questo punto is aggiungevano colonne che si pensa possano
essere significative come giorno/mese/anno/hr/weekday nel vecchio
dataset ma nel nuovo in realtà non lo si fa poichè molto di questi dati
vengono già dati gratuitamente. Vedremo alla fine dell' analisi che
effettivamente risulteranno utili ma questo lo si può immaginare anche a
monte poichè dei consumatori possono seguire un pattern per scegliere il
giorno in cui comprare(il giorno successivo al giorno di paga). A questo
punto il dataset viene diviso in customers che hanno speso e quelli che
non lo hanno fatto poichè poi dividendo in dev e val poi vogliamo
mantenere la proporzione.

\subsection{Funzione encode\_cats}\label{funzione-encode_cats}

Questa funzione rappresenta il cuore dell' encoding e la sua forza, come
in tutte le altre funzioni di questo notebook, è fornire un interfaccia
generica, potente, efficinete e trasparente. Da notare infatti il
parametro target che permette di definire parametricamente le colonne
che \emph{non} si vuole modificare. Da notare inoltre che la funzione
prende il train e il test e li encoda in contemporanea per mantenerne la
coerenza : \((train[cols] \cap test[cols]) - targetValue = \emptyset\)
dove targetValue è il valore che vogliamo predire alla fine. La funzione
encoda correttamente i valori ma non soltanto... dopo l' encoding si
perdono i nomi delle colonne a causa della trasformazione in
numpy.matrix ma non è detto che la prima colonna del test encoded sia la
stessa del train encoded, sembra complicato ma il fatto è che un volta
che spacchetto una colonna e la faccio diventare 50 colonne se quella
colonna conteneva un solo valore diverso nel test bè allora avrò colonne
diverse e magari pure un ordine diverso nelle colonne, ricordandomi
anche che non so più i nomi delle nuove colonne. Il problema quindi
viene risolto droppando le colonne diverse tra i due nuovi datasets
encodati e sfruttando il fatto che i nomi delle colonne possono essere
restituiti dall' oggetto che ha prodotto l' encoding(non è così semplice
perchè anche i nomi sono una matrice quindi si passa per un flattening
intermedio).

\subsection{Funzione group\_me}\label{funzione-group_me}

Questa seconda funzione viene subito dopo l' encoding e punta a rendere
i dati, già ben preparati, più parlanti rispetto al singolo utente che è
il nostro target. Si decide di fare la media dei dati numerici poichè
non abbiamo un encoding su quelli ma abbiamo solo bisogno di un range in
cui farli variare e si prende la moda dei dati categoriali, cioè si
prende per ogni utente il valore che appare più volte poichè non
servirebbero a nulla dei valori coninui se stiamo usando valori
discreti: non verrebbero mappati! Il calcolo della moda inoltre è un
punto dolente dell' intero processo poichè richiede più tempo del
previsto(infatti si penserà ad un ottimizzazione o ad una sostituzione
con un funzionale simile ma più efficiente come la media che purtroppo
restituirà valori continui ma questo fortunatamente viene sopportato da
un albero di regressione) questo perchè richiede di aggregare tutti i
dati e calcolare la frequenza di ogni valore per poi scegliere il valore
con più ripetizioni.

\subsection{Conclusioni}\label{conclusioni}

L' ultima parte di meno importanza ma comunque necessaria è la divisione
del dataset in dev e val del train per poter allenare un regressore
lineare e un albero di regressione, qui inoltre si fa
stratified-sampling dividendo il dataset per spesa cioè i compratori
vengono divisi dai visitatori non compratori in maniera da prendere la
stessa percentuale da tutti e due i gruppi per poter mantenere il
rapporto dell' 1\% di cui parla la legge di Pareto. In questa sezione
vengono fatte poi operazioni minime di preparazione del dataset
scegliendo le colonne di training, la percentuale di righe per training
e test e poco altro.

    \section{Training e Scoring}\label{training-e-scoring}

\begin{center}\rule{0.5\linewidth}{\linethickness}\end{center}

\subsection{Introduzione}\label{introduzione}

Sono stati fatti vari tentativi per ottenere un buon punteggio e si è
iniziato prevedendo la media della spesa degli utenti, poi si è passati
a prevedere con una regressione lineare che però correttamente otteneva
lo stesso score di un file zero-filled e alla fine si è compreso che la
scelta migliore trattandosi di un task di regressione che poteva però
contare su tanti dati e di diversa natura e non solo numerici sarebbe
stato usare alberi di regressione. Si è inoltre tentato di cercare
clusters attraverso i dati ma si è notato che comparando il revenue con
la feature hits(che ha massima correlazione) si ottiene un insieme
globulare di punti nello zero(disponibili le immagini pubblicamente sul
mio account plotly "riccardobernardi"). Inoltre trattandosi di un task
di regressione l' idea di fare clustering è stata abbandonata poichè non
abbastanza orientata al risultato(erano già state fatte molte EDA su
kaggle). Il training avviene attraverso un framework molto diffuso nelle
competizioni di ML chiamato LightGBM, viene scelto questo framework
perchè ben documentato, potente, coerente coi tipi di dati e mantenuto
da un ente che punta su questo framework. Il training avviene (per ora)
solo attraverso un dataset di dev cioè di training e di val cioè di
validation (ma non si esclude di aggiungere una k-fold validation oppure
di usare cv già incluso nell' engine di LGBM) che vengono estratti dal
training iniziale mantenendo la proporzione dell' 1\% dei compratori.
Questo perchè ci potrebbe essere un alto rischio di mandare in val solo
zeri e di avere un modello che predice solo zeri. Tutto ciò si chiama
stratified sampling.

\subsection{Lin-reg}\label{lin-reg}

Per poter dare dimostrazione delle scelte fatte si vuole dare la
possibilità di testare una predizione fatta con LinReg piuttosto che con
un albero di regressione. La regressione lineare è stata scelta perchè
intrinsecamente il task è di regressione cioè richiede che la predizione
sia un valore continuo. Alla fine la regressione sarà solo usata come
cartina tornasole/benchmark per verificare che l' albero di regressione
superi la regressione lineare, il che è molto verosimile. La regressione
lineare è implementata usando la libreria scikit-learn. Alla fine
comunque si potrà notare che la regressione lineare funzionerà
egregiamente, tanto da superare addirittura gli alberi di regressione
per casi speciali. Si fa però notare che questo accade solo nel nuovo
dataset mentre invece nel vecchio dataset l' albero di regressione non
aveva pari.

\subsection{LightGBM}\label{lightgbm}

La conoscenza viene alla fine estratta tramite questa libreria
sviluappata da microsoft che implementa alberi di regressione cioè
alberi che dividono in maniera da aumentare l' information gain o
diminuire l' impurità, ecco di seguito le possibili implementazioni
spiegate:

Gini Index:

Si calcola l' impurità dei dati in ingresso

\[Gini(D) = 1 - \sum(p_i)^2\]

Si calcola l' impurità per ogni valore A della feature presa in
considerazione e poi si ripeterà per ogni feature nel dataset di
training. Viene simulato lo split binario e si calcola se sia
conveniente per ogni feature così da scegliere la feature che migliora
l' impurità.

\[Gini_{A}(D) = \frac{|D_1|}{|D|}\cdot Gini(D_1) + \frac{|D_2|}{|D|}\cdot Gini(D_2) \]

Si calcola poi l' impurità dello split simulato rispettivamente al
non-split:

\[Gini(D) - Gini_{A}(D_j)\]

E questo quindi è un valore da rendere minimo per scegliere il miglior
split per creare un nodo che sia localmente ottimo e cioè che abbia
buone probabilità di creare una foglia pura.

oppure

Information Gain di Shannon

Si calcola l' entropia per il dataset non splittato

\[Info(D) = -\sum(p_i)\cdot log_2(p_i)\]

poi si calcola l' entropia per ogni valore A della feature presa in
considerazione e come prima poi si vorrà ripeterlo per ogni feature per
poi decidere quale feature determina il migliore split

\[Info_A(D) = \frac{|D_j|}{|D|}\cdot Info(D_j)\]

Si calcola poi il gain rispetto al non-split del dataset

\[Gain(A) = Info(D) - Info_A(D)\]

Ed in questo caso sarà un valore da massimizzare perchè diversamente dal
gini poichè qui abbiamo il gain e non l' impurità

inoltre

Gain Ratio:

\[SplitInfo_A(D) = -\sum \frac{|D_j|}{|D|}\cdot log_2(\frac{|D_j|}{|D|})\]

\[GainRatio(A) = \frac{Gain(A)}{SplitInfo_A(D)}\]

Il vantaggio di usare un albero di regressione è che i dati vengono
divisi in sotto-nodi a sx o dx rispetto alle formule qui sopra per
semplicemente rendere minimo l' rmse, in questa maniera saremo sicuri
che dopo un certo numero di iterazioni avremo una granularità molto fine
nella scelta dei valori di previsione. E' infatti dimostrato che
continuando a dividere in maniera iterativa si otterrà il miglior valore
possibile nonostante sia una scelta greedy. La varianza dei dati viene
migliorata per esempio mediando i valori tra le foglie o tra i fari
alberi nel caso di una random forest. Per quanto invece il bias esso
viene ridotto usando il metodo del boosting cioè re-samplando i dati più
rari per ottenere un migliore train su di essi. Bisogna inoltre
guardarsi dall' overfitting contro il quale si è più volte andati
incontro con questo dataset ma lo si è evitato utilizzando il metodo del
bagging e usando random forests.

\subsection{Conclusioni}\label{conclusioni}

Si sono implementati metodi per il tracking e il monitoring dei
risultati su file csv in maniera da poter decidere i metodi migliori da
utilizzare dopo un po di tentativi. Si è inoltre prodotto codice che
tenta iterativamente tutte le configurazioni ragionevoli in un certo
range di valori per poter plottare un grafico che indichi qual' è il
miglior valore da scegliere per una certa feature

    \section{Scelte}\label{scelte}

\begin{center}\rule{0.5\linewidth}{\linethickness}\end{center}

Durante lo svolgimento del progetto sono state fatte scelte che si
desidera puntualizzare qui

\subsection{(No) EDA}\label{no-eda}

Si sceglie di non dare nella versione finale nessuno sketch di EDA per
due motivi : il primo è che in realtà l' EDA è stato fatto ma quando il
codice è diventato preponderante non era più possibile runnare EDA e
codice nello stesso file perchè la grafica provoca rallentamenti e
rendeva difficile il debug visivamente, l' EDA quindi è disponibile in
parte presso il mio account di plotly pubblicamente ("riccardobernardi")
e in maggior parte nelle versioni precedenti di questo kernel e nel file
"extra.py" in questa cartella. Il secondo motivo è che ad un certo punto
dello sviluppo si è ritenuto che l' EDA fosse troppo semplicistico e non
fosse abbastanza fare analisi colonna per colonna trovandosi a lavorare
con centinaia di colonne. Si fa notare che comunque tutti gli EDA
sufficienti a farsi un ottima idea del dataset sono disponibili
pubblicamente sui kernel di kaggle di questa competizione.

\subsection{transactionRevenue e
totalTransactionRevenue}\label{transactionrevenue-e-totaltransactionrevenue}

Si è deciso in di seguire i consigli del docente nel togliere il
transactionRevenue e prevedere il totalTransactionRevenue senza quel
dato, si vuole puntualizzare che in questa maniera il risultato risulta
ben peggiore rispetto alla LB di Kaggle in cui si ha ottenuto un ottimo
score di \textbf{0.1227} per questo motivo si lascia la possbilità di
attivare o meno la funzionalità che elimina transactionRevenue all
inizio del kernel kaggle-v6.0 per avere l' ebbrezza di un punteggio
molto molto alto(rmse molto basso). Detto ciò si è assolutamente d'
accordo sul fatto che eliminare transactionRevenue sia la cosa più
corretta da fare poichè non si può contare sul fatto che poi col nuovo
test sia ancora una colonna disponibile(transactionRevenue è molto
predittiva per totalTransactionRevenue e lo si può vedere dal grafico
della feat importance alla fine del kernel principale)

EDIT:Da un recente comunicato di Kaggle hanno dichiarato il campo
transactionRevenue deprecato

\subsection{Inefficienza della moda}\label{inefficienza-della-moda}

Si prevede di fare una funzione meno sematicamente significativa della
moda come la media o una approssimazione della moda per raggruppare i
valori categorici di ogni utente a seguito del fatto che allo stato
attuale rappresenta il collo di bottiglia di tutto il progetto in
termini di tempo, persino meno performante del OneHotEncoding.

\subsection{Rischio di overfit con
productSKU}\label{rischio-di-overfit-con-productsku}

Si è deciso di non utilizzare i codici prodotti per migliorare le
previsioni poichè l' elaborazione risulterebbe complessa a fronte di un
possibile overfit ma non si esclude di poter trovare una maniera di
estrapolare conoscenza dagli stessi in altre maniere con magari delle
euristiche sulle tipologie di prodotti

\subsection{Molte feature sintetiche vs poche features
reali}\label{molte-feature-sintetiche-vs-poche-features-reali}

Si è scelto, ascoltando i consigli del docente, ad un certo punto dell'
analisi del dataset, di creare features sintetiche piuttosto che di
analizzare singolarmente la moltitudine di dati presenti. Si è fatto ciò
poichè in questo corso si vuole privilegiare il fatto che i pattern sui
dati escano spontaneamente e non si hanno sempre delle asserzioni sulle
distribuzioni dei dati.

\subsection{Cross-validation}\label{cross-validation}

Si è scelto(per ora ma non si esclude di aggiungerlo) di non aggiungere
la cross-validazione/k-fold-validazione dei dati puntando maggiormente
sullo sviluppo di features più importanti poichè già di per sè l'
elaborazione del dataset è piuttosto pesante e i dati non mancano.
Avendo tanti dati si può pensare che basti solo prendere qualche decina
di migliaia di righe in più per ottenere un migliore score.

    \section{Features aggiuntive}\label{features-aggiuntive}

\begin{center}\rule{0.5\linewidth}{\linethickness}\end{center}

Si vuole ora dare una breve lista sulle features disponibili in questa
toolbox:

\begin{itemize}
\item
  si può fare un rapido ed efficace tuning dei parametri modificandoli
  all' inizio del notebook, il vantaggio di farlo lì è oltre all'
  accentramento di tutti i parametri più importanti anche il fatto che
  quei parametri verranno dumpati insieme allo score ottenuto su un file
  e letti dalla tabella qui sotto per avere subito idea del
  miglioramento o meno
\item
  utilizzo di pandas
\item
  è disponibile una progressbar che viene già usata ma può essere
  istanziata in qualsiasi momento anche lato utente
\item
  è disponibile una funzione chiamata cc(parameters) che è già presente
  in parti del codice e in maniera silente performa una forte verifica
  sulla correttezza e consistenza del dataset, questo permette di
  ottenere subito un errore appena una funzione che modifica il dataset
  crea inconsistenze, in tale maniera si eviteranno errori a cascata
\item
  viene utilizzata l' API di kaggle per submittare perciò non si fa mai
  la fatica di muovere il mouse
\item
  tutte le variabili che regolano il notebook sono in una dashboard
  iniziale contenuti in un dict chiamato parameters che istanzia le
  coppie (k:v) all inizio del programma inserendole nelle variabili, il
  vantaggio è automatizzare la stampa dei parametri insieme allo score
  nel file tests.csv che compare qui sotto
\item
  le operazioni più costose hanno \%time davanti così da poter sapere
  dopo un po di tentativi il tempo che si aspetterà
\item
  l' operazione più costosa (group\_me) può essere approssimata passando
  da "mode" a "mean"
\item
  le funzioni più importanti o costose danno un output minimo, le meno
  importanti e in tempo \(\Theta(1)\) non danno output a meno che non
  lavorino sul filesystem
\item
  viene data la possibilità di salvare il dataset su disco nel caso in
  cui si usi la moda poichè è un operazione che impiega molto tempo
  mentre in tutti gli altri casi si consiglia di runnare normalmente il
  kernel poichè media/mediana/moda\_approssimata hanno ottima
  complessità
\item
  testing iterativo dei parametri per scegliere il migliore valore per
  ciascuno di essi(per ora solo n\_leaves ma assolutamente scalabile a
  piacere con l' aggiunta di un for che cicla la lista di parametri da
  testare per migliorare lo score)
\end{itemize}

    \section{Features future}\label{features-future}

\begin{center}\rule{0.5\linewidth}{\linethickness}\end{center}

\begin{itemize}
\tightlist
\item
  Si vorrebbe innanzitutto fare un merge del test e del train poichè
  dopo il data-leakage essi sono entrambi datasets di test ma per ora
  non lo si è fatto poichè già il training viene fatto su una piccola
  parte del dataset train
\item
  pca sui numerics
\item
  aggiungere weekday
\item
  cascading = fare n predittori specializzati per cose diverse su set
  diversi
\item
  probabilmente le azioni di un utente dipende da quello che ha fatto al
  massimo nelle 2 settimane precedenti
\end{itemize}

    \section{I numeri di questa
competizione}\label{i-numeri-di-questa-competizione}

\begin{center}\rule{0.5\linewidth}{\linethickness}\end{center}

\begin{itemize}
\item
  Score committato : 0.1227
\item
  numero di submissions totali : 120 in circa 2 mesi(tempo
  effettivamente disponibile) -\textgreater{} una media di circa 2 al
  giorno, notare che il numero di submissions è limitato a 5 al giorno
  perciò l' impegno è stato costante
\item
  6 versioni del kernel per un unica competizione(verrà consegnata solo
  la versione definitiva ma sono disponibili anche le altre)
\item
  6 moduli separati di funzioni per un totale di più di 1200 righe di
  codice in totale(solo nei moduli)
\item
  44 commits in 27 giorni, poco meno di 2 commits al giorno(in continua
  aggiunta, inoltre il github non è stato fatto da subito ma solo alla
  versione 6 poichè ritenuta matura)
\item
  2 settimane per adattare il nuovo dataset al vecchio kernel. Il must
  era mantenere più informazioni possibili e scartare quasi nulla.
\item
  168hr il tempo impiegato in questo progetto, la stima è fatta per
  difetto e seguendo l' idea del problema di fermi: ogni 5 submissions
  si pensa che la prima costi 7hr(nella versione 6 del kernel cioè
  quest' ultima il kernel veniva preparato e le submissions venivano
  fatte il giorno dopo quindi ben più di 7hr) e che le successive
  vengano gratuitamente perciò
  \(\frac{120}{5}\cdot 7hr = 24 \cdot 7 = 168\)
\item
  4, il livello del mio account su kaggle cioè competition contributor.
\end{itemize}

    \section{Considerazioni}\label{considerazioni}

\begin{center}\rule{0.5\linewidth}{\linethickness}\end{center}

Si è notato che la regressione lineare è più efficiente e più rapida di
quanto testato inizialmente nei kernel di versioni precedenti che non
godevano di OHE e addirittura supera LightGBM in caso di dataset ridotti
cioè inferiori a 100000 righe, questo probabilmente perchè fintanto che
il dataset è ridotto e ci sono molti zeri la regressione lineare
indovina sempre puntando tutto a 0 mentre l' albero di regressione con
pochi dati si può confondere, tanto più si allarga il dataset tanto più
la regressione lineare riduce la sua precisione. Inoltre introducendo
transactionRevenue l' albero di regressione supera di molto la
regressione lineare anche per datasets piccoli. Inoltre se si aggiunge
il transaction revenue LightGBM riesce molto meglio della regressione
lineare.

    \section{Dati raccolti}\label{dati-raccolti}

\begin{center}\rule{0.5\linewidth}{\linethickness}\end{center}

Avvertenze:

\begin{itemize}
\item
  nella tabella qui sotto trovare il numero di righe nel test a 500 non
  è un errore! essendo questi dati interni e non spediti a kaggle posso
  diminuire le righe di test(cioè da prevedere) per potermi concentrare
  su ottenere dati nel train. I dati raccolti perciò sono rispetto al
  validation.
\item
  il final\_score è calcolato senza il transactionRevenue come
  consigliato dal docente ma è stato inserito un parametro per sfruttare
  o meno questo vantaggio
\item
  i parametri vengono scritti in maniera automatica sul file perciò si
  avrà cura di tenere a mente che se il "tipo" == "lin\_reg" i parametri
  che regolano LightGBM non devono essere valutati
\end{itemize}

Considerazioni:

\begin{itemize}
\item
  Si è voluto sperimentare come l' accuratezza migliorasse all'
  aumentare dei dati, non solo per migliorare il punteggio ma per
  controllare che la LGN valesse o meglio per avere riscontro della
  convergenza degli algoritmi usati. Se così non fosse stato allora
  avremmo scoperto errori nella programmazione.
\item
  Si è notato che col nuovo dataset si è in grado di fare previsioni
  migliori che non col dataset precedente fornito da kaggle(disponibili
  i kernel di versione precedente a questa a dimostrarlo, score
  \textgreater{}= 1.57 circa)
\item
  In maniera abbastanza inasspettata si è scoperto che la regressione
  lineare dà uno score migliore di un albero di regressione nel caso in
  cui la transactionRevenue venga esclusa e il dataset sia piccolo, è
  facile immaginare sia così poichè il dataset ha molti zeri.
\item
  Si fa notare che dove \(|dataset| \geqslant 10000\) si preferisce
  usare la media per i dati categoriali invece della moda poichè con un
  dataset di 10000 righe la moda impiega 34min di computazione e per un
  dataset maggiore sarebbe impraticabile. E' perciò un approssimazione
  non voluta ma obbligata, detto ciò il risultato è comunque molto
  valido, sia con la moda su un dataset piccolo sia con la media con un
  dataset grande e questo è supportato da fatto che nonostante la media
  non sia significativa per dati categoriale questo viene comunque
  accettato da un albero di regressione(cambieranno soltanto i valori
  nei nodi)
\end{itemize}

    \begin{Verbatim}[commandchars=\\\{\}]
{\color{incolor}In [{\color{incolor}1}]:} \PY{k+kn}{import} \PY{n+nn}{pandas} \PY{k}{as} \PY{n+nn}{pd}
        \PY{n}{df} \PY{o}{=} \PY{n}{pd}\PY{o}{.}\PY{n}{read\PYZus{}csv}\PY{p}{(}\PY{l+s+s2}{\PYZdq{}}\PY{l+s+s2}{tests.csv}\PY{l+s+s2}{\PYZdq{}}\PY{p}{)}
\end{Verbatim}


    \begin{Verbatim}[commandchars=\\\{\}]
{\color{incolor}In [{\color{incolor}2}]:} \PY{n}{pd}\PY{o}{.}\PY{n}{read\PYZus{}csv}\PY{p}{(}\PY{l+s+s2}{\PYZdq{}}\PY{l+s+s2}{tests.csv}\PY{l+s+s2}{\PYZdq{}}\PY{p}{)}
\end{Verbatim}


\begin{Verbatim}[commandchars=\\\{\}]
{\color{outcolor}Out[{\color{outcolor}2}]:}      max\_new\_feat  commit  n\_leaves  feature\_fraction  bagging\_fraction  \textbackslash{}
        0              50       0       100              0.90              0.80   
        1              50       0       100              0.90              0.80   
        2              50       0      8192              0.80              0.95   
        3              50       0      8192              0.40              0.25   
        4              50       0       100              0.70              0.50   
        5              50       0       200              0.99              0.99   
        6              50       0      8192              0.40              0.25   
        7              50       0      1000              0.99              0.99   
        8              50       0      8192              0.99              0.99   
        9             500       0       100              0.99              0.99   
        10            500       0       100              0.99              0.99   
        11            500       0      8192              0.99              0.99   
        12            500       0       100              0.99              0.99   
        13            500       0       100              0.99              0.99   
        14            500       0      8192              0.99              0.99   
        15            500       0      8192              0.99              0.99   
        16            500       0       100              0.99              0.99   
        17            500       0       100              0.99              0.99   
        18            500       0      8192              0.99              0.99   
        19            500       0       100              0.99              0.99   
        20            500       0       100              0.99              0.99   
        21            500       0      8192              0.99              0.99   
        22            500       0      4096              0.99              0.99   
        23            500       0      4096              0.99              0.99   
        24            500       0      2048              0.99              0.99   
        25            500       0      1024              0.99              0.99   
        26            500       0      1024              0.99              0.99   
        27            500       0      2048              0.99              0.99   
        28            500       0      1024              0.99              0.99   
        29            500       0       512              0.99              0.99   
        ..            {\ldots}     {\ldots}       {\ldots}               {\ldots}               {\ldots}   
        167           500       0        32              0.99              0.99   
        168           500       0        16              0.99              0.99   
        169           500       0        32              0.99              0.99   
        170           500       0        64              0.99              0.99   
        171           500       0       128              0.99              0.99   
        172           500       0       256              0.99              0.99   
        173           500       0       512              0.99              0.99   
        174           500       0        16              0.99              0.99   
        175           500       0        32              0.99              0.99   
        176           500       0        64              0.99              0.99   
        177           500       0       128              0.99              0.99   
        178           500       0       256              0.99              0.99   
        179           500       0       512              0.99              0.99   
        180           500       0        16              0.99              0.99   
        181           500       0        17              0.99              0.99   
        182           500       0        18              0.99              0.99   
        183           500       0        19              0.99              0.99   
        184           500       0        20              0.99              0.99   
        185           500       0        21              0.99              0.99   
        186           500       0        22              0.99              0.99   
        187           500       0        23              0.99              0.99   
        188           500       0        24              0.99              0.99   
        189           500       0        25              0.99              0.99   
        190           500       0        26              0.99              0.99   
        191           500       0        27              0.99              0.99   
        192           500       0        28              0.99              0.99   
        193           500       0        29              0.99              0.99   
        194           500       0        30              0.99              0.99   
        195           500       0        31              0.99              0.99   
        196           500       0        32              0.99              0.99   
        
             learn\_rate  train\_rows  test\_rows  test\_also\_lin\_reg  bagging\_freq  \textbackslash{}
        0         0.004      100000        500                  1             1   
        1         0.004      100000        500                  1             1   
        2         0.004      100000        500                  1            20   
        3         0.004      100000        500                  1             1   
        4         0.004      100000        500                  1             1   
        5         0.004      100000        500                  1             1   
        6         0.004      100000        500                  1             1   
        7         0.004      100000        500                  1             1   
        8         0.004      100000        500                  1            20   
        9         0.004      100000        500                  1             1   
        10        0.004      100000        500                  1             1   
        11        0.004      100000        500                  1            20   
        12        0.004       10000        500                  1             1   
        13        0.004       10000        500                  1             1   
        14        0.004       10000        500                  1            20   
        15        0.004       10000        500                  1            20   
        16        0.004       10000        500                  1             1   
        17        0.004       10000        500                  1             1   
        18        0.004       10000        500                  1            20   
        19        0.004      100000        500                  1             1   
        20        0.004      100000        500                  1             1   
        21        0.004      100000        500                  1            20   
        22        0.004      100000        500                  1            20   
        23        0.004      100000        500                  1            20   
        24        0.004      100000        500                  1            20   
        25        0.004      100000        500                  1             1   
        26        0.004      100000        500                  1             1   
        27        0.004      100000        500                  1            20   
        28        0.004      100000        500                  1            20   
        29        0.004      100000        500                  1            20   
        ..          {\ldots}         {\ldots}        {\ldots}                {\ldots}           {\ldots}   
        167       0.004       10000        500                  1            20   
        168       0.004       10000        500                  1            20   
        169       0.004       10000        500                  1            20   
        170       0.004       10000        500                  1            20   
        171       0.004       10000        500                  1            20   
        172       0.004       10000        500                  1            20   
        173       0.004       10000        500                  1            20   
        174       0.004       10000        500                  1            20   
        175       0.004       10000        500                  1            20   
        176       0.004       10000        500                  1            20   
        177       0.004       10000        500                  1            20   
        178       0.004       10000        500                  1            20   
        179       0.004       10000        500                  1            20   
        180       0.004       10000        500                  1            20   
        181       0.004       10000        500                  1            20   
        182       0.004       10000        500                  1            20   
        183       0.004       10000        500                  1            20   
        184       0.004       10000        500                  1            20   
        185       0.004       10000        500                  1            20   
        186       0.004       10000        500                  1            20   
        187       0.004       10000        500                  1            20   
        188       0.004       10000        500                  1            20   
        189       0.004       10000        500                  1            20   
        190       0.004       10000        500                  1            20   
        191       0.004       10000        500                  1            20   
        192       0.004       10000        500                  1            20   
        193       0.004       10000        500                  1            20   
        194       0.004       10000        500                  1            20   
        195       0.004       10000        500                  1            20   
        196       0.004       10000        500                  1            20   
        
             transactionRevenue  percentage grouping\_mode\_cats  final\_score  \textbackslash{}
        0                     0          18               mode     1.276707   
        1                     0          18               mode     1.517199   
        2                     0          18               mode     1.530116   
        3                     0          18               mode     1.536360   
        4                     0          18               mode     1.506410   
        5                     0          18               mode     1.534567   
        6                     0          18               mode     1.536360   
        7                     0          18               mode     1.534808   
        8                     0          18               mode     1.536360   
        9                     1          18               mode     1.056175   
        10                    1          18               mode     0.034409   
        11                    1          18               mode     0.145676   
        12                    0          18               mode     1.092369   
        13                    0          18               mode     1.002389   
        14                    0          18               mode     1.002389   
        15                    0          18               mode     1.055687   
        16                    0          18               mean     1.222760   
        17                    0          18               mean     1.639784   
        18                    0          18               mean     1.824215   
        19                    0          18               mean     1.287251   
        20                    0          18               mean     1.525817   
        21                    0          18               mean     1.646259   
        22                    0          18               mean     1.646259   
        23                    0          18               mean     1.646259   
        24                    0          18               mean     1.646259   
        25                    0          18               mean     1.282734   
        26                    0          18               mean     1.532604   
        27                    0          18               mean     1.665875   
        28                    0          18               mean     1.665875   
        29                    0          18               mean     1.665875   
        ..                  {\ldots}         {\ldots}                {\ldots}          {\ldots}   
        167                   0          18               mode     1.006207   
        168                   0          18               mode     1.010126   
        169                   0          18               mode     1.010126   
        170                   0          18               mode     1.010126   
        171                   0          18               mode     1.010126   
        172                   0          18               mode     1.010126   
        173                   0          18               mode     1.010126   
        174                   0          18               mode     1.010126   
        175                   0          18               mode     1.010126   
        176                   0          18               mode     1.010126   
        177                   0          18               mode     1.010126   
        178                   0          18               mode     1.010126   
        179                   0          18               mode     1.010126   
        180                   0          18               mode     1.010126   
        181                   0          18               mode     1.010126   
        182                   0          18               mode     1.010126   
        183                   0          18               mode     1.010126   
        184                   0          18               mode     1.010126   
        185                   0          18               mode     1.010126   
        186                   0          18               mode     1.010126   
        187                   0          18               mode     1.010126   
        188                   0          18               mode     1.010126   
        189                   0          18               mode     1.010126   
        190                   0          18               mode     1.010126   
        191                   0          18               mode     1.010126   
        192                   0          18               mode     1.010126   
        193                   0          18               mode     1.010126   
        194                   0          18               mode     1.010126   
        195                   0          18               mode     1.010126   
        196                   0          18               mode     1.010126   
        
             min\_child\_samples         type  
        0                   -1      lin\_reg  
        1                   -1     LightGBM  
        2                   30  LightGBM\_rf  
        3                   10  LightGBM\_rf  
        4                   -1     LightGBM  
        5                   -1     LightGBM  
        6                   10  LightGBM\_rf  
        7                   -1     LightGBM  
        8                   10  LightGBM\_rf  
        9                   -1      lin\_reg  
        10                  -1     LightGBM  
        11                  10  LightGBM\_rf  
        12                  -1      lin\_reg  
        13                  -1     LightGBM  
        14                  10  LightGBM\_rf  
        15                  10  LightGBM\_rf  
        16                  -1      lin\_reg  
        17                  -1     LightGBM  
        18                  10  LightGBM\_rf  
        19                  -1      lin\_reg  
        20                  -1     LightGBM  
        21                  10  LightGBM\_rf  
        22                  10  LightGBM\_rf  
        23                  10  LightGBM\_rf  
        24                  10  LightGBM\_rf  
        25                  -1      lin\_reg  
        26                  -1     LightGBM  
        27                  10  LightGBM\_rf  
        28                  10  LightGBM\_rf  
        29                  10  LightGBM\_rf  
        ..                 {\ldots}          {\ldots}  
        167                 10     LightGBM  
        168                 10     LightGBM  
        169                 10     LightGBM  
        170                 10     LightGBM  
        171                 10     LightGBM  
        172                 10     LightGBM  
        173                 10     LightGBM  
        174                 10     LightGBM  
        175                 10     LightGBM  
        176                 10     LightGBM  
        177                 10     LightGBM  
        178                 10     LightGBM  
        179                 10     LightGBM  
        180                 10     LightGBM  
        181                 10     LightGBM  
        182                 10     LightGBM  
        183                 10     LightGBM  
        184                 10     LightGBM  
        185                 10     LightGBM  
        186                 10     LightGBM  
        187                 10     LightGBM  
        188                 10     LightGBM  
        189                 10     LightGBM  
        190                 10     LightGBM  
        191                 10     LightGBM  
        192                 10     LightGBM  
        193                 10     LightGBM  
        194                 10     LightGBM  
        195                 10     LightGBM  
        196                 10     LightGBM  
        
        [197 rows x 16 columns]
\end{Verbatim}
            
    \section{Esperimenti}\label{esperimenti}

\begin{center}\rule{0.5\linewidth}{\linethickness}\end{center}

Da qui in poi si è voluto sperimentare come diversi valori dei parametri
potessero cambiare lo score e ovviamente poi si è reportato tutto qui
sotto. Sono chiamati esperimenti poichè data un ipotesi sono stati
interrogati i dati per vedere se l' ipotesi fosse supportata da essi ed
ogni esperimento porta con sè una "morale" o conclusione per confermare
o smentire l' ipotesi.

Si avvisa che le rappresentazioni nel grafico sono tante volte
manipolate attraverso alcune traformazioni di funzioni/normalizzazioni
per rendere evidente il risultato ottenuto

    \subsection{Esperimento 1:}\label{esperimento-1}

\begin{center}\rule{0.5\linewidth}{\linethickness}\end{center}

\subsubsection{Aumentare le foglie in una random forest migliora lo
score in virtù della teoria che ci sta
dietro}\label{aumentare-le-foglie-in-una-random-forest-migliora-lo-score-in-virtuxf9-della-teoria-che-ci-sta-dietro}

Sapendo la teoria delle random forest e cioè che si vuole portare i vari
alberi all' overfit per poi mediarne i risultati si è pensato che
aumentando il numero di foglie potesse aumentare la precisione

Si è effettivamente notato dai dati raccolti che aumentare le foglie
diminuisce l' RMSE cioè migliora lo score.

Lo si può vedere nel grafico qui sotto dove sono stati normalizzati i
dati per permettere di vedere il rapporto che hanno i due. I dati sono
stati scelti tra campioni simili per esempio cercando campioni con
stesso numero di righe poichè quello è un fattore determinante per lo
score

    \begin{Verbatim}[commandchars=\\\{\}]
{\color{incolor}In [{\color{incolor}4}]:} \PY{k+kn}{import} \PY{n+nn}{matplotlib}\PY{n+nn}{.}\PY{n+nn}{pyplot} \PY{k}{as} \PY{n+nn}{plt}
        \PY{k+kn}{from} \PY{n+nn}{generic} \PY{k}{import} \PY{n}{norm}
        
        \PY{n}{fig}\PY{p}{,} \PY{n}{ax} \PY{o}{=} \PY{n}{plt}\PY{o}{.}\PY{n}{subplots}\PY{p}{(} \PY{n}{figsize}\PY{o}{=}\PY{p}{(}\PY{l+m+mi}{9}\PY{p}{,}\PY{l+m+mi}{3}\PY{p}{)}\PY{p}{,} \PY{n}{tight\PYZus{}layout}\PY{o}{=}\PY{k+kc}{True}\PY{p}{)}
        
        \PY{n}{arr} \PY{o}{=} \PY{p}{(}\PY{n}{df}\PY{p}{[}\PY{l+s+s2}{\PYZdq{}}\PY{l+s+s2}{type}\PY{l+s+s2}{\PYZdq{}}\PY{p}{]} \PY{o}{==} \PY{l+s+s2}{\PYZdq{}}\PY{l+s+s2}{LightGBM\PYZus{}rf}\PY{l+s+s2}{\PYZdq{}}\PY{p}{)} \PY{o}{\PYZam{}} \PY{p}{(}\PY{n}{df}\PY{p}{[}\PY{l+s+s2}{\PYZdq{}}\PY{l+s+s2}{train\PYZus{}rows}\PY{l+s+s2}{\PYZdq{}}\PY{p}{]} \PY{o}{==} \PY{l+m+mi}{100000}\PY{p}{)} \PY{o}{\PYZam{}} \PY{p}{(}\PY{n}{df}\PY{p}{[}\PY{l+s+s2}{\PYZdq{}}\PY{l+s+s2}{max\PYZus{}new\PYZus{}feat}\PY{l+s+s2}{\PYZdq{}}\PY{p}{]} \PY{o}{==} \PY{l+m+mi}{500}\PY{p}{)} \PY{o}{\PYZam{}} \PY{p}{(}\PY{n}{df}\PY{p}{[}\PY{l+s+s2}{\PYZdq{}}\PY{l+s+s2}{transactionRevenue}\PY{l+s+s2}{\PYZdq{}}\PY{p}{]} \PY{o}{==} \PY{l+m+mi}{0}\PY{p}{)}
        \PY{n}{l} \PY{o}{=} \PY{n}{df}\PY{o}{.}\PY{n}{loc}\PY{p}{[}\PY{n}{arr}\PY{p}{,}\PY{l+s+s2}{\PYZdq{}}\PY{l+s+s2}{n\PYZus{}leaves}\PY{l+s+s2}{\PYZdq{}}\PY{p}{]}
        \PY{n}{ax}\PY{o}{.}\PY{n}{plot}\PY{p}{(} \PY{n}{norm}\PY{p}{(}\PY{n}{l}\PY{p}{)} \PY{p}{,}\PY{l+s+s1}{\PYZsq{}}\PY{l+s+s1}{r\PYZhy{}*}\PY{l+s+s1}{\PYZsq{}}\PY{p}{)}
        \PY{c+c1}{\PYZsh{}foglie in rosso}
        \PY{c+c1}{\PYZsh{}al diminuire delle foglie il punteggio cala cioè peggiora}
        
        \PY{n}{l} \PY{o}{=} \PY{n}{df}\PY{o}{.}\PY{n}{loc}\PY{p}{[}\PY{n}{arr}\PY{p}{,}\PY{l+s+s2}{\PYZdq{}}\PY{l+s+s2}{final\PYZus{}score}\PY{l+s+s2}{\PYZdq{}}\PY{p}{]}
        \PY{n}{ax}\PY{o}{.}\PY{n}{plot}\PY{p}{(} \PY{n}{norm}\PY{p}{(}\PY{n}{l}\PY{p}{)} \PY{p}{)}
\end{Verbatim}


\begin{Verbatim}[commandchars=\\\{\}]
{\color{outcolor}Out[{\color{outcolor}4}]:} [<matplotlib.lines.Line2D at 0x1a1c9fffd0>]
\end{Verbatim}
            
    \begin{center}
    \adjustimage{max size={0.9\linewidth}{0.9\paperheight}}{output_17_1.png}
    \end{center}
    { \hspace*{\fill} \\}
    
    \begin{Verbatim}[commandchars=\\\{\}]
{\color{incolor}In [{\color{incolor}5}]:} \PY{n}{arr} \PY{o}{=} \PY{p}{(}\PY{n}{df}\PY{p}{[}\PY{l+s+s2}{\PYZdq{}}\PY{l+s+s2}{type}\PY{l+s+s2}{\PYZdq{}}\PY{p}{]} \PY{o}{==} \PY{l+s+s2}{\PYZdq{}}\PY{l+s+s2}{LightGBM\PYZus{}rf}\PY{l+s+s2}{\PYZdq{}}\PY{p}{)} \PY{o}{\PYZam{}} \PY{p}{(}\PY{n}{df}\PY{p}{[}\PY{l+s+s2}{\PYZdq{}}\PY{l+s+s2}{train\PYZus{}rows}\PY{l+s+s2}{\PYZdq{}}\PY{p}{]} \PY{o}{==} \PY{l+m+mi}{100000}\PY{p}{)} \PY{o}{\PYZam{}} \PY{p}{(}\PY{n}{df}\PY{p}{[}\PY{l+s+s2}{\PYZdq{}}\PY{l+s+s2}{max\PYZus{}new\PYZus{}feat}\PY{l+s+s2}{\PYZdq{}}\PY{p}{]} \PY{o}{==} \PY{l+m+mi}{500}\PY{p}{)} \PY{o}{\PYZam{}} \PY{p}{(}\PY{n}{df}\PY{p}{[}\PY{l+s+s2}{\PYZdq{}}\PY{l+s+s2}{transactionRevenue}\PY{l+s+s2}{\PYZdq{}}\PY{p}{]} \PY{o}{==} \PY{l+m+mi}{0}\PY{p}{)}
        \PY{n}{l} \PY{o}{=} \PY{n}{df}\PY{o}{.}\PY{n}{loc}\PY{p}{[}\PY{n}{arr}\PY{p}{,}\PY{l+s+s2}{\PYZdq{}}\PY{l+s+s2}{n\PYZus{}leaves}\PY{l+s+s2}{\PYZdq{}}\PY{p}{]}
        
        
        \PY{n}{l} \PY{o}{=} \PY{n}{df}\PY{o}{.}\PY{n}{loc}\PY{p}{[}\PY{n}{arr}\PY{p}{,}\PY{l+s+s2}{\PYZdq{}}\PY{l+s+s2}{final\PYZus{}score}\PY{l+s+s2}{\PYZdq{}}\PY{p}{]}
        \PY{n}{m} \PY{o}{=} \PY{n+nb}{min}\PY{p}{(}\PY{n}{l}\PY{p}{)}
        \PY{n}{m}
        
        \PY{n}{df}\PY{o}{.}\PY{n}{loc}\PY{p}{[}\PY{n}{df}\PY{p}{[}\PY{l+s+s2}{\PYZdq{}}\PY{l+s+s2}{final\PYZus{}score}\PY{l+s+s2}{\PYZdq{}}\PY{p}{]} \PY{o}{==} \PY{n}{m}\PY{p}{,}\PY{l+s+s2}{\PYZdq{}}\PY{l+s+s2}{n\PYZus{}leaves}\PY{l+s+s2}{\PYZdq{}}\PY{p}{]}
\end{Verbatim}


\begin{Verbatim}[commandchars=\\\{\}]
{\color{outcolor}Out[{\color{outcolor}5}]:} 34    16
        Name: n\_leaves, dtype: int64
\end{Verbatim}
            
    \subsubsection{Conclusione inaspettata}\label{conclusione-inaspettata}

Aumentare le foglie è utile ma in questo caso ha provocato un punteggio
più scarso. Il risultato è sicuramente di rilievo e si fornisce la
scelta migliore in questo caso: la riga 34 del dataset df contiene i
valori chiave per ottenere il miglior risultato nel caso di questo
esperimento cioè 16 foglie

    \subsection{Esperimento 2}\label{esperimento-2}

\begin{center}\rule{0.5\linewidth}{\linethickness}\end{center}

\subsubsection{su piccoli dataset la LR performa meglio di
LightGBM(senza
transactionRevenue)}\label{su-piccoli-dataset-la-lr-performa-meglio-di-lightgbmsenza-transactionrevenue}

Inizialmente si pensava che LR non potesse in alcuna maniera superare in
score LGBM ma stando ai dati su piccoli datasets(\(\tilde{}10000\)) la
LR performa meglio. Si può pensare che LR predica sempre 0.0 che su
questo dataset è il 99\% dei valori del totalTransactionRevenue perciò
faccia un ottimo score mentre si può pensare che LGBM non abbia
abbastanza dati per fare un buon training e che fare boosting porti solo
ad un forte overfit per l' ovvio basso supporto da parte dei dati.
Questo fatto infatti non sarà più vero su datasets più grandi

    \begin{Verbatim}[commandchars=\\\{\}]
{\color{incolor}In [{\color{incolor}6}]:} \PY{k+kn}{import} \PY{n+nn}{matplotlib}\PY{n+nn}{.}\PY{n+nn}{pyplot} \PY{k}{as} \PY{n+nn}{plt}
        
        \PY{n}{fig}\PY{p}{,} \PY{n}{ax} \PY{o}{=} \PY{n}{plt}\PY{o}{.}\PY{n}{subplots}\PY{p}{(} \PY{n}{figsize}\PY{o}{=}\PY{p}{(}\PY{l+m+mi}{9}\PY{p}{,}\PY{l+m+mi}{3}\PY{p}{)}\PY{p}{,} \PY{n}{tight\PYZus{}layout}\PY{o}{=}\PY{k+kc}{True}\PY{p}{)}
        
        \PY{n}{arr} \PY{o}{=} \PY{p}{(}\PY{n}{df}\PY{p}{[}\PY{l+s+s2}{\PYZdq{}}\PY{l+s+s2}{type}\PY{l+s+s2}{\PYZdq{}}\PY{p}{]} \PY{o}{==} \PY{l+s+s2}{\PYZdq{}}\PY{l+s+s2}{lin\PYZus{}reg}\PY{l+s+s2}{\PYZdq{}}\PY{p}{)} \PY{o}{\PYZam{}} \PY{p}{(}\PY{n}{df}\PY{p}{[}\PY{l+s+s2}{\PYZdq{}}\PY{l+s+s2}{train\PYZus{}rows}\PY{l+s+s2}{\PYZdq{}}\PY{p}{]} \PY{o}{==} \PY{l+m+mi}{100000}\PY{p}{)} \PY{o}{\PYZam{}} \PY{p}{(}\PY{n}{df}\PY{p}{[}\PY{l+s+s2}{\PYZdq{}}\PY{l+s+s2}{max\PYZus{}new\PYZus{}feat}\PY{l+s+s2}{\PYZdq{}}\PY{p}{]} \PY{o}{==} \PY{l+m+mi}{500}\PY{p}{)} \PY{o}{\PYZam{}} \PY{p}{(}\PY{n}{df}\PY{p}{[}\PY{l+s+s2}{\PYZdq{}}\PY{l+s+s2}{transactionRevenue}\PY{l+s+s2}{\PYZdq{}}\PY{p}{]} \PY{o}{==} \PY{l+m+mi}{0}\PY{p}{)} \PY{o}{\PYZam{}} \PY{p}{(}\PY{n}{df}\PY{p}{[}\PY{l+s+s2}{\PYZdq{}}\PY{l+s+s2}{grouping\PYZus{}mode\PYZus{}cats}\PY{l+s+s2}{\PYZdq{}}\PY{p}{]} \PY{o}{==} \PY{l+s+s2}{\PYZdq{}}\PY{l+s+s2}{mean}\PY{l+s+s2}{\PYZdq{}}\PY{p}{)}
        \PY{n}{l} \PY{o}{=} \PY{n}{df}\PY{o}{.}\PY{n}{loc}\PY{p}{[}\PY{n}{arr}\PY{p}{,}\PY{l+s+s2}{\PYZdq{}}\PY{l+s+s2}{final\PYZus{}score}\PY{l+s+s2}{\PYZdq{}}\PY{p}{]}
        \PY{n}{ax}\PY{o}{.}\PY{n}{plot}\PY{p}{(} \PY{l+m+mi}{1}\PY{o}{/}\PY{n}{l} \PY{p}{)}
        
        \PY{n}{arr} \PY{o}{=} \PY{p}{(}\PY{n}{df}\PY{p}{[}\PY{l+s+s2}{\PYZdq{}}\PY{l+s+s2}{type}\PY{l+s+s2}{\PYZdq{}}\PY{p}{]} \PY{o}{==} \PY{l+s+s2}{\PYZdq{}}\PY{l+s+s2}{LightGBM}\PY{l+s+s2}{\PYZdq{}}\PY{p}{)} \PY{o}{\PYZam{}} \PY{p}{(}\PY{n}{df}\PY{p}{[}\PY{l+s+s2}{\PYZdq{}}\PY{l+s+s2}{train\PYZus{}rows}\PY{l+s+s2}{\PYZdq{}}\PY{p}{]} \PY{o}{==} \PY{l+m+mi}{100000}\PY{p}{)} \PY{o}{\PYZam{}} \PY{p}{(}\PY{n}{df}\PY{p}{[}\PY{l+s+s2}{\PYZdq{}}\PY{l+s+s2}{max\PYZus{}new\PYZus{}feat}\PY{l+s+s2}{\PYZdq{}}\PY{p}{]} \PY{o}{==} \PY{l+m+mi}{500}\PY{p}{)} \PY{o}{\PYZam{}} \PY{p}{(}\PY{n}{df}\PY{p}{[}\PY{l+s+s2}{\PYZdq{}}\PY{l+s+s2}{transactionRevenue}\PY{l+s+s2}{\PYZdq{}}\PY{p}{]} \PY{o}{==} \PY{l+m+mi}{0}\PY{p}{)} \PY{o}{\PYZam{}} \PY{p}{(}\PY{n}{df}\PY{p}{[}\PY{l+s+s2}{\PYZdq{}}\PY{l+s+s2}{grouping\PYZus{}mode\PYZus{}cats}\PY{l+s+s2}{\PYZdq{}}\PY{p}{]} \PY{o}{==} \PY{l+s+s2}{\PYZdq{}}\PY{l+s+s2}{mean}\PY{l+s+s2}{\PYZdq{}}\PY{p}{)}
        \PY{n}{l} \PY{o}{=} \PY{n}{df}\PY{o}{.}\PY{n}{loc}\PY{p}{[}\PY{n}{arr}\PY{p}{,}\PY{l+s+s2}{\PYZdq{}}\PY{l+s+s2}{final\PYZus{}score}\PY{l+s+s2}{\PYZdq{}}\PY{p}{]}
        \PY{n}{ax}\PY{o}{.}\PY{n}{plot}\PY{p}{(} \PY{l+m+mi}{1}\PY{o}{/}\PY{n}{l} \PY{p}{)}
        
        \PY{n}{arr} \PY{o}{=} \PY{p}{(}\PY{n}{df}\PY{p}{[}\PY{l+s+s2}{\PYZdq{}}\PY{l+s+s2}{type}\PY{l+s+s2}{\PYZdq{}}\PY{p}{]} \PY{o}{==} \PY{l+s+s2}{\PYZdq{}}\PY{l+s+s2}{LightGBM\PYZus{}rf}\PY{l+s+s2}{\PYZdq{}}\PY{p}{)} \PY{o}{\PYZam{}} \PY{p}{(}\PY{n}{df}\PY{p}{[}\PY{l+s+s2}{\PYZdq{}}\PY{l+s+s2}{train\PYZus{}rows}\PY{l+s+s2}{\PYZdq{}}\PY{p}{]} \PY{o}{==} \PY{l+m+mi}{100000}\PY{p}{)} \PY{o}{\PYZam{}} \PY{p}{(}\PY{n}{df}\PY{p}{[}\PY{l+s+s2}{\PYZdq{}}\PY{l+s+s2}{max\PYZus{}new\PYZus{}feat}\PY{l+s+s2}{\PYZdq{}}\PY{p}{]} \PY{o}{==} \PY{l+m+mi}{500}\PY{p}{)} \PY{o}{\PYZam{}} \PY{p}{(}\PY{n}{df}\PY{p}{[}\PY{l+s+s2}{\PYZdq{}}\PY{l+s+s2}{transactionRevenue}\PY{l+s+s2}{\PYZdq{}}\PY{p}{]} \PY{o}{==} \PY{l+m+mi}{0}\PY{p}{)} \PY{o}{\PYZam{}} \PY{p}{(}\PY{n}{df}\PY{p}{[}\PY{l+s+s2}{\PYZdq{}}\PY{l+s+s2}{grouping\PYZus{}mode\PYZus{}cats}\PY{l+s+s2}{\PYZdq{}}\PY{p}{]} \PY{o}{==} \PY{l+s+s2}{\PYZdq{}}\PY{l+s+s2}{mean}\PY{l+s+s2}{\PYZdq{}}\PY{p}{)}
        \PY{n}{l} \PY{o}{=} \PY{n}{df}\PY{o}{.}\PY{n}{loc}\PY{p}{[}\PY{n}{arr}\PY{p}{,}\PY{l+s+s2}{\PYZdq{}}\PY{l+s+s2}{final\PYZus{}score}\PY{l+s+s2}{\PYZdq{}}\PY{p}{]}
        \PY{n}{ax}\PY{o}{.}\PY{n}{plot}\PY{p}{(} \PY{l+m+mi}{1}\PY{o}{/}\PY{n}{l} \PY{p}{)}
        \PY{n}{ax}\PY{o}{.}\PY{n}{legend}\PY{p}{(}\PY{p}{[}\PY{l+s+s2}{\PYZdq{}}\PY{l+s+s2}{LR}\PY{l+s+s2}{\PYZdq{}}\PY{p}{,}\PY{l+s+s2}{\PYZdq{}}\PY{l+s+s2}{LGBM}\PY{l+s+s2}{\PYZdq{}}\PY{p}{,}\PY{l+s+s2}{\PYZdq{}}\PY{l+s+s2}{LGBMRF}\PY{l+s+s2}{\PYZdq{}}\PY{p}{]}\PY{p}{)}
\end{Verbatim}


\begin{Verbatim}[commandchars=\\\{\}]
{\color{outcolor}Out[{\color{outcolor}6}]:} <matplotlib.legend.Legend at 0x1a1cc71898>
\end{Verbatim}
            
    \begin{center}
    \adjustimage{max size={0.9\linewidth}{0.9\paperheight}}{output_21_1.png}
    \end{center}
    { \hspace*{\fill} \\}
    
    \begin{Verbatim}[commandchars=\\\{\}]
{\color{incolor}In [{\color{incolor}7}]:} \PY{n}{arr} \PY{o}{=} \PY{p}{(}\PY{n}{df}\PY{p}{[}\PY{l+s+s2}{\PYZdq{}}\PY{l+s+s2}{train\PYZus{}rows}\PY{l+s+s2}{\PYZdq{}}\PY{p}{]} \PY{o}{==} \PY{l+m+mi}{100000}\PY{p}{)} \PY{o}{\PYZam{}} \PY{p}{(}\PY{n}{df}\PY{p}{[}\PY{l+s+s2}{\PYZdq{}}\PY{l+s+s2}{max\PYZus{}new\PYZus{}feat}\PY{l+s+s2}{\PYZdq{}}\PY{p}{]} \PY{o}{==} \PY{l+m+mi}{500}\PY{p}{)} \PY{o}{\PYZam{}} \PY{p}{(}\PY{n}{df}\PY{p}{[}\PY{l+s+s2}{\PYZdq{}}\PY{l+s+s2}{transactionRevenue}\PY{l+s+s2}{\PYZdq{}}\PY{p}{]} \PY{o}{==} \PY{l+m+mi}{0}\PY{p}{)} \PY{o}{\PYZam{}} \PY{p}{(}\PY{n}{df}\PY{p}{[}\PY{l+s+s2}{\PYZdq{}}\PY{l+s+s2}{grouping\PYZus{}mode\PYZus{}cats}\PY{l+s+s2}{\PYZdq{}}\PY{p}{]} \PY{o}{==} \PY{l+s+s2}{\PYZdq{}}\PY{l+s+s2}{mean}\PY{l+s+s2}{\PYZdq{}}\PY{p}{)}
        
        \PY{n}{l} \PY{o}{=} \PY{n}{df}\PY{o}{.}\PY{n}{loc}\PY{p}{[}\PY{n}{arr}\PY{p}{,}\PY{l+s+s2}{\PYZdq{}}\PY{l+s+s2}{final\PYZus{}score}\PY{l+s+s2}{\PYZdq{}}\PY{p}{]}
        \PY{n}{m} \PY{o}{=} \PY{n+nb}{min}\PY{p}{(}\PY{n}{l}\PY{p}{)}
        \PY{n}{m}
        
        \PY{n}{df}\PY{o}{.}\PY{n}{loc}\PY{p}{[}\PY{n}{df}\PY{p}{[}\PY{l+s+s2}{\PYZdq{}}\PY{l+s+s2}{final\PYZus{}score}\PY{l+s+s2}{\PYZdq{}}\PY{p}{]} \PY{o}{==} \PY{n}{m}\PY{p}{,}\PY{l+s+s2}{\PYZdq{}}\PY{l+s+s2}{type}\PY{l+s+s2}{\PYZdq{}}\PY{p}{]}
\end{Verbatim}


\begin{Verbatim}[commandchars=\\\{\}]
{\color{outcolor}Out[{\color{outcolor}7}]:} 39    lin\_reg
        Name: type, dtype: object
\end{Verbatim}
            
    \subsubsection{Conclusione}\label{conclusione}

L' affermazione di questo esperimento viene confermata dai dati

    \subsection{Esperimento 3}\label{esperimento-3}

\begin{center}\rule{0.5\linewidth}{\linethickness}\end{center}

\subsubsection{su piccoli dataset LightGBM con singolo albero non ha
eguali(con
transactionRevenue)}\label{su-piccoli-dataset-lightgbm-con-singolo-albero-non-ha-egualicon-transactionrevenue}

Inizialmente si pensava che LightGBM\_rf cioè non il singolo albero ma
la random forest potesse performare meglio del singolo albero ma
osservando i dati si può giungere alla conclusione che LGBM a singolo
albero che sfrutta il 100\% delle features performa meglio sia della RF
sia della LR. Si può probabilmente asserire che LGBM riesca a battere in
quanto a score la RF poichè appunto il singolo albero dispone di tutte
le features mentre nella RF le feats vengono divise sui vari alberi
della foresta e questo porti a molti alberi con molti valori sballati ed
essendo la media un funzionale non robusto questo porti ad uno score
peggiore.

Si noti che a differenza degli altri esperimenti qui si è mantenuto il
transactionRevenue come predittore poichè si vuole dimostrare che nella
competizione di kaggle si è fatta una scelta sensata volta ad ottenere
il punteggio maggiore possibile in LB

    \begin{Verbatim}[commandchars=\\\{\}]
{\color{incolor}In [{\color{incolor}8}]:} \PY{k+kn}{import} \PY{n+nn}{matplotlib}\PY{n+nn}{.}\PY{n+nn}{pyplot} \PY{k}{as} \PY{n+nn}{plt}
        
        \PY{n}{fig}\PY{p}{,} \PY{n}{ax} \PY{o}{=} \PY{n}{plt}\PY{o}{.}\PY{n}{subplots}\PY{p}{(} \PY{n}{figsize}\PY{o}{=}\PY{p}{(}\PY{l+m+mi}{9}\PY{p}{,}\PY{l+m+mi}{3}\PY{p}{)}\PY{p}{,} \PY{n}{tight\PYZus{}layout}\PY{o}{=}\PY{k+kc}{True}\PY{p}{)}
        
        \PY{n}{arr} \PY{o}{=} \PY{p}{(}\PY{n}{df}\PY{p}{[}\PY{l+s+s2}{\PYZdq{}}\PY{l+s+s2}{type}\PY{l+s+s2}{\PYZdq{}}\PY{p}{]} \PY{o}{==} \PY{l+s+s2}{\PYZdq{}}\PY{l+s+s2}{lin\PYZus{}reg}\PY{l+s+s2}{\PYZdq{}}\PY{p}{)} \PY{o}{\PYZam{}} \PY{p}{(}\PY{n}{df}\PY{p}{[}\PY{l+s+s2}{\PYZdq{}}\PY{l+s+s2}{train\PYZus{}rows}\PY{l+s+s2}{\PYZdq{}}\PY{p}{]} \PY{o}{==} \PY{l+m+mi}{100000}\PY{p}{)} \PY{o}{\PYZam{}} \PY{p}{(}\PY{n}{df}\PY{p}{[}\PY{l+s+s2}{\PYZdq{}}\PY{l+s+s2}{max\PYZus{}new\PYZus{}feat}\PY{l+s+s2}{\PYZdq{}}\PY{p}{]} \PY{o}{==} \PY{l+m+mi}{500}\PY{p}{)}
        \PY{n}{l} \PY{o}{=} \PY{n}{df}\PY{o}{.}\PY{n}{loc}\PY{p}{[}\PY{n}{arr}\PY{p}{,}\PY{l+s+s2}{\PYZdq{}}\PY{l+s+s2}{final\PYZus{}score}\PY{l+s+s2}{\PYZdq{}}\PY{p}{]}
        \PY{n}{ax}\PY{o}{.}\PY{n}{plot}\PY{p}{(} \PY{l+m+mi}{1}\PY{o}{/}\PY{n}{l} \PY{p}{)}
        
        \PY{n}{arr} \PY{o}{=} \PY{p}{(}\PY{n}{df}\PY{p}{[}\PY{l+s+s2}{\PYZdq{}}\PY{l+s+s2}{type}\PY{l+s+s2}{\PYZdq{}}\PY{p}{]} \PY{o}{==} \PY{l+s+s2}{\PYZdq{}}\PY{l+s+s2}{LightGBM}\PY{l+s+s2}{\PYZdq{}}\PY{p}{)} \PY{o}{\PYZam{}} \PY{p}{(}\PY{n}{df}\PY{p}{[}\PY{l+s+s2}{\PYZdq{}}\PY{l+s+s2}{train\PYZus{}rows}\PY{l+s+s2}{\PYZdq{}}\PY{p}{]} \PY{o}{==} \PY{l+m+mi}{100000}\PY{p}{)} \PY{o}{\PYZam{}} \PY{p}{(}\PY{n}{df}\PY{p}{[}\PY{l+s+s2}{\PYZdq{}}\PY{l+s+s2}{max\PYZus{}new\PYZus{}feat}\PY{l+s+s2}{\PYZdq{}}\PY{p}{]} \PY{o}{==} \PY{l+m+mi}{500}\PY{p}{)}
        \PY{n}{l} \PY{o}{=} \PY{n}{df}\PY{o}{.}\PY{n}{loc}\PY{p}{[}\PY{n}{arr}\PY{p}{,}\PY{l+s+s2}{\PYZdq{}}\PY{l+s+s2}{final\PYZus{}score}\PY{l+s+s2}{\PYZdq{}}\PY{p}{]}
        \PY{n}{ax}\PY{o}{.}\PY{n}{plot}\PY{p}{(} \PY{l+m+mi}{1}\PY{o}{/}\PY{n}{l} \PY{p}{)}
        
        \PY{n}{arr} \PY{o}{=} \PY{p}{(}\PY{n}{df}\PY{p}{[}\PY{l+s+s2}{\PYZdq{}}\PY{l+s+s2}{type}\PY{l+s+s2}{\PYZdq{}}\PY{p}{]} \PY{o}{==} \PY{l+s+s2}{\PYZdq{}}\PY{l+s+s2}{LightGBM\PYZus{}rf}\PY{l+s+s2}{\PYZdq{}}\PY{p}{)} \PY{o}{\PYZam{}} \PY{p}{(}\PY{n}{df}\PY{p}{[}\PY{l+s+s2}{\PYZdq{}}\PY{l+s+s2}{train\PYZus{}rows}\PY{l+s+s2}{\PYZdq{}}\PY{p}{]} \PY{o}{==} \PY{l+m+mi}{100000}\PY{p}{)} \PY{o}{\PYZam{}} \PY{p}{(}\PY{n}{df}\PY{p}{[}\PY{l+s+s2}{\PYZdq{}}\PY{l+s+s2}{max\PYZus{}new\PYZus{}feat}\PY{l+s+s2}{\PYZdq{}}\PY{p}{]} \PY{o}{==} \PY{l+m+mi}{500}\PY{p}{)}
        \PY{n}{l} \PY{o}{=} \PY{n}{df}\PY{o}{.}\PY{n}{loc}\PY{p}{[}\PY{n}{arr}\PY{p}{,}\PY{l+s+s2}{\PYZdq{}}\PY{l+s+s2}{final\PYZus{}score}\PY{l+s+s2}{\PYZdq{}}\PY{p}{]}
        \PY{n}{ax}\PY{o}{.}\PY{n}{plot}\PY{p}{(} \PY{l+m+mi}{1}\PY{o}{/}\PY{n}{l} \PY{p}{)}
        \PY{n}{ax}\PY{o}{.}\PY{n}{legend}\PY{p}{(}\PY{p}{[}\PY{l+s+s2}{\PYZdq{}}\PY{l+s+s2}{LR}\PY{l+s+s2}{\PYZdq{}}\PY{p}{,}\PY{l+s+s2}{\PYZdq{}}\PY{l+s+s2}{LGBM}\PY{l+s+s2}{\PYZdq{}}\PY{p}{,}\PY{l+s+s2}{\PYZdq{}}\PY{l+s+s2}{LGBMRF}\PY{l+s+s2}{\PYZdq{}}\PY{p}{]}\PY{p}{)}
\end{Verbatim}


\begin{Verbatim}[commandchars=\\\{\}]
{\color{outcolor}Out[{\color{outcolor}8}]:} <matplotlib.legend.Legend at 0x1a1ccb2400>
\end{Verbatim}
            
    \begin{center}
    \adjustimage{max size={0.9\linewidth}{0.9\paperheight}}{output_25_1.png}
    \end{center}
    { \hspace*{\fill} \\}
    
    \subsubsection{Conclusione}\label{conclusione}

L' affermazione di questo esperimento viene confermata dai dati

    \subsection{Esperimento 4}\label{esperimento-4}

\begin{center}\rule{0.5\linewidth}{\linethickness}\end{center}

\subsubsection{miglior n\_leaves per
LGBM}\label{miglior-n_leaves-per-lgbm}

Si sta cercando il miglior valore per il numero di foglie per il singolo
albero di regressione. Si cerca il valore tentando iterativamente
potenze del 2 e poi selezionando il minimo rmse.

    \begin{Verbatim}[commandchars=\\\{\}]
{\color{incolor}In [{\color{incolor}9}]:} \PY{k+kn}{import} \PY{n+nn}{matplotlib}\PY{n+nn}{.}\PY{n+nn}{pyplot} \PY{k}{as} \PY{n+nn}{plt}
        \PY{k+kn}{from} \PY{n+nn}{generic} \PY{k}{import} \PY{n}{norm}
        
        \PY{n}{fig}\PY{p}{,} \PY{n}{ax} \PY{o}{=} \PY{n}{plt}\PY{o}{.}\PY{n}{subplots}\PY{p}{(} \PY{n}{figsize}\PY{o}{=}\PY{p}{(}\PY{l+m+mi}{9}\PY{p}{,}\PY{l+m+mi}{3}\PY{p}{)}\PY{p}{,} \PY{n}{tight\PYZus{}layout}\PY{o}{=}\PY{k+kc}{True}\PY{p}{)}
        
        \PY{n}{arr} \PY{o}{=} \PY{p}{(}\PY{n}{df}\PY{p}{[}\PY{l+s+s2}{\PYZdq{}}\PY{l+s+s2}{type}\PY{l+s+s2}{\PYZdq{}}\PY{p}{]} \PY{o}{==} \PY{l+s+s2}{\PYZdq{}}\PY{l+s+s2}{LightGBM}\PY{l+s+s2}{\PYZdq{}}\PY{p}{)} \PY{o}{\PYZam{}} \PY{p}{(}\PY{n}{df}\PY{p}{[}\PY{l+s+s2}{\PYZdq{}}\PY{l+s+s2}{train\PYZus{}rows}\PY{l+s+s2}{\PYZdq{}}\PY{p}{]} \PY{o}{==} \PY{l+m+mi}{100000}\PY{p}{)} \PY{o}{\PYZam{}} \PY{p}{(}\PY{n}{df}\PY{p}{[}\PY{l+s+s2}{\PYZdq{}}\PY{l+s+s2}{max\PYZus{}new\PYZus{}feat}\PY{l+s+s2}{\PYZdq{}}\PY{p}{]} \PY{o}{==} \PY{l+m+mi}{500}\PY{p}{)} \PY{o}{\PYZam{}} \PY{p}{(}\PY{n}{df}\PY{p}{[}\PY{l+s+s2}{\PYZdq{}}\PY{l+s+s2}{transactionRevenue}\PY{l+s+s2}{\PYZdq{}}\PY{p}{]} \PY{o}{==} \PY{l+m+mi}{0}\PY{p}{)}
        \PY{n}{l} \PY{o}{=} \PY{n}{df}\PY{o}{.}\PY{n}{loc}\PY{p}{[}\PY{n}{arr}\PY{p}{,}\PY{l+s+s2}{\PYZdq{}}\PY{l+s+s2}{n\PYZus{}leaves}\PY{l+s+s2}{\PYZdq{}}\PY{p}{]}
        \PY{n}{ax}\PY{o}{.}\PY{n}{plot}\PY{p}{(} \PY{n}{norm}\PY{p}{(}\PY{n}{l}\PY{p}{)} \PY{p}{,}\PY{l+s+s1}{\PYZsq{}}\PY{l+s+s1}{r\PYZhy{}*}\PY{l+s+s1}{\PYZsq{}}\PY{p}{)}
        \PY{c+c1}{\PYZsh{}foglie in rosso}
        \PY{c+c1}{\PYZsh{}al diminuire delle foglie il punteggio cala cioè peggiora}
        
        \PY{n}{l} \PY{o}{=} \PY{n}{df}\PY{o}{.}\PY{n}{loc}\PY{p}{[}\PY{n}{arr}\PY{p}{,}\PY{l+s+s2}{\PYZdq{}}\PY{l+s+s2}{final\PYZus{}score}\PY{l+s+s2}{\PYZdq{}}\PY{p}{]}
        \PY{n}{ax}\PY{o}{.}\PY{n}{plot}\PY{p}{(} \PY{n}{norm}\PY{p}{(}\PY{n}{l}\PY{p}{)} \PY{p}{)}
\end{Verbatim}


\begin{Verbatim}[commandchars=\\\{\}]
{\color{outcolor}Out[{\color{outcolor}9}]:} [<matplotlib.lines.Line2D at 0x1a1cddd7b8>]
\end{Verbatim}
            
    \begin{center}
    \adjustimage{max size={0.9\linewidth}{0.9\paperheight}}{output_28_1.png}
    \end{center}
    { \hspace*{\fill} \\}
    
    \begin{Verbatim}[commandchars=\\\{\}]
{\color{incolor}In [{\color{incolor}10}]:} \PY{n}{arr} \PY{o}{=} \PY{p}{(}\PY{n}{df}\PY{p}{[}\PY{l+s+s2}{\PYZdq{}}\PY{l+s+s2}{type}\PY{l+s+s2}{\PYZdq{}}\PY{p}{]} \PY{o}{==} \PY{l+s+s2}{\PYZdq{}}\PY{l+s+s2}{LightGBM}\PY{l+s+s2}{\PYZdq{}}\PY{p}{)} \PY{o}{\PYZam{}} \PY{p}{(}\PY{n}{df}\PY{p}{[}\PY{l+s+s2}{\PYZdq{}}\PY{l+s+s2}{train\PYZus{}rows}\PY{l+s+s2}{\PYZdq{}}\PY{p}{]} \PY{o}{==} \PY{l+m+mi}{100000}\PY{p}{)} \PY{o}{\PYZam{}} \PY{p}{(}\PY{n}{df}\PY{p}{[}\PY{l+s+s2}{\PYZdq{}}\PY{l+s+s2}{max\PYZus{}new\PYZus{}feat}\PY{l+s+s2}{\PYZdq{}}\PY{p}{]} \PY{o}{==} \PY{l+m+mi}{500}\PY{p}{)} \PY{o}{\PYZam{}} \PY{p}{(}\PY{n}{df}\PY{p}{[}\PY{l+s+s2}{\PYZdq{}}\PY{l+s+s2}{transactionRevenue}\PY{l+s+s2}{\PYZdq{}}\PY{p}{]} \PY{o}{==} \PY{l+m+mi}{0}\PY{p}{)}
         \PY{n}{l} \PY{o}{=} \PY{n}{df}\PY{o}{.}\PY{n}{loc}\PY{p}{[}\PY{n}{arr}\PY{p}{,}\PY{l+s+s2}{\PYZdq{}}\PY{l+s+s2}{n\PYZus{}leaves}\PY{l+s+s2}{\PYZdq{}}\PY{p}{]}
         
         
         \PY{n}{l} \PY{o}{=} \PY{n}{df}\PY{o}{.}\PY{n}{loc}\PY{p}{[}\PY{n}{arr}\PY{p}{,}\PY{l+s+s2}{\PYZdq{}}\PY{l+s+s2}{final\PYZus{}score}\PY{l+s+s2}{\PYZdq{}}\PY{p}{]}
         \PY{n}{m} \PY{o}{=} \PY{n+nb}{min}\PY{p}{(}\PY{n}{l}\PY{p}{)}
         \PY{n}{m}
         
         \PY{n}{df}\PY{o}{.}\PY{n}{loc}\PY{p}{[}\PY{n}{df}\PY{p}{[}\PY{l+s+s2}{\PYZdq{}}\PY{l+s+s2}{final\PYZus{}score}\PY{l+s+s2}{\PYZdq{}}\PY{p}{]} \PY{o}{==} \PY{n}{m}\PY{p}{,}\PY{l+s+s2}{\PYZdq{}}\PY{l+s+s2}{n\PYZus{}leaves}\PY{l+s+s2}{\PYZdq{}}\PY{p}{]}
\end{Verbatim}


\begin{Verbatim}[commandchars=\\\{\}]
{\color{outcolor}Out[{\color{outcolor}10}]:} 52    20
         Name: n\_leaves, dtype: int64
\end{Verbatim}
            
    \subsubsection{Conclusione}\label{conclusione}

Concludo che la scelta migliore per il numero di foglie con LGBM è 20
per l' esperimento descritto qui sopra

    \subsection{Esperimento 5}\label{esperimento-5}

\begin{center}\rule{0.5\linewidth}{\linethickness}\end{center}

\subsubsection{la moda per le cats è più potente(e sensata) della
media}\label{la-moda-per-le-cats-uxe8-piuxf9-potentee-sensata-della-media}

Si vorrebbe scegliere sempre di utilizzare la moda per i dati
categoriali poichè non ha senso usare la media su valori che sono
discreti(e.g.: dire che ho metà iphone e metà samsung quando entro in un
negozio...), il problema è che la moda ha un costo computazionale troppo
alto. Si vuole perciò dimostrare dati alla mano che usare la moda è
comunque una scelta forte e ottima se solo fosse possibile computarla in
maniera più rapida.

    \begin{Verbatim}[commandchars=\\\{\}]
{\color{incolor}In [{\color{incolor}11}]:} \PY{k+kn}{import} \PY{n+nn}{matplotlib}\PY{n+nn}{.}\PY{n+nn}{pyplot} \PY{k}{as} \PY{n+nn}{plt}
         \PY{k+kn}{from} \PY{n+nn}{generic} \PY{k}{import} \PY{n}{norm}
         
         \PY{n}{fig}\PY{p}{,} \PY{n}{ax} \PY{o}{=} \PY{n}{plt}\PY{o}{.}\PY{n}{subplots}\PY{p}{(} \PY{n}{figsize}\PY{o}{=}\PY{p}{(}\PY{l+m+mi}{9}\PY{p}{,}\PY{l+m+mi}{3}\PY{p}{)}\PY{p}{,} \PY{n}{tight\PYZus{}layout}\PY{o}{=}\PY{k+kc}{True}\PY{p}{)}
         
         \PY{n}{arr} \PY{o}{=} \PY{p}{(}\PY{n}{df}\PY{p}{[}\PY{l+s+s2}{\PYZdq{}}\PY{l+s+s2}{grouping\PYZus{}mode\PYZus{}cats}\PY{l+s+s2}{\PYZdq{}}\PY{p}{]} \PY{o}{==} \PY{l+s+s2}{\PYZdq{}}\PY{l+s+s2}{mode}\PY{l+s+s2}{\PYZdq{}}\PY{p}{)} \PY{o}{\PYZam{}} \PY{p}{(}\PY{n}{df}\PY{p}{[}\PY{l+s+s2}{\PYZdq{}}\PY{l+s+s2}{train\PYZus{}rows}\PY{l+s+s2}{\PYZdq{}}\PY{p}{]} \PY{o}{==} \PY{l+m+mi}{100000}\PY{p}{)} \PY{o}{\PYZam{}} \PY{p}{(}\PY{n}{df}\PY{p}{[}\PY{l+s+s2}{\PYZdq{}}\PY{l+s+s2}{max\PYZus{}new\PYZus{}feat}\PY{l+s+s2}{\PYZdq{}}\PY{p}{]} \PY{o}{==} \PY{l+m+mi}{500}\PY{p}{)}
         \PY{n}{l} \PY{o}{=} \PY{n}{df}\PY{o}{.}\PY{n}{loc}\PY{p}{[}\PY{n}{arr}\PY{p}{,}\PY{l+s+s2}{\PYZdq{}}\PY{l+s+s2}{final\PYZus{}score}\PY{l+s+s2}{\PYZdq{}}\PY{p}{]}
         \PY{n}{ax}\PY{o}{.}\PY{n}{plot}\PY{p}{(} \PY{n}{norm}\PY{p}{(}\PY{n}{l}\PY{p}{)} \PY{p}{)}
         
         \PY{n}{arr} \PY{o}{=} \PY{p}{(}\PY{n}{df}\PY{p}{[}\PY{l+s+s2}{\PYZdq{}}\PY{l+s+s2}{grouping\PYZus{}mode\PYZus{}cats}\PY{l+s+s2}{\PYZdq{}}\PY{p}{]} \PY{o}{==} \PY{l+s+s2}{\PYZdq{}}\PY{l+s+s2}{mean}\PY{l+s+s2}{\PYZdq{}}\PY{p}{)} \PY{o}{\PYZam{}} \PY{p}{(}\PY{n}{df}\PY{p}{[}\PY{l+s+s2}{\PYZdq{}}\PY{l+s+s2}{train\PYZus{}rows}\PY{l+s+s2}{\PYZdq{}}\PY{p}{]} \PY{o}{==} \PY{l+m+mi}{100000}\PY{p}{)} \PY{o}{\PYZam{}} \PY{p}{(}\PY{n}{df}\PY{p}{[}\PY{l+s+s2}{\PYZdq{}}\PY{l+s+s2}{max\PYZus{}new\PYZus{}feat}\PY{l+s+s2}{\PYZdq{}}\PY{p}{]} \PY{o}{==} \PY{l+m+mi}{500}\PY{p}{)}
         \PY{n}{l} \PY{o}{=} \PY{n}{df}\PY{o}{.}\PY{n}{loc}\PY{p}{[}\PY{n}{arr}\PY{p}{,}\PY{l+s+s2}{\PYZdq{}}\PY{l+s+s2}{final\PYZus{}score}\PY{l+s+s2}{\PYZdq{}}\PY{p}{]}
         \PY{n}{ax}\PY{o}{.}\PY{n}{plot}\PY{p}{(} \PY{n}{norm}\PY{p}{(}\PY{n}{l}\PY{p}{)} \PY{p}{)}
         
         \PY{n}{ax}\PY{o}{.}\PY{n}{legend}\PY{p}{(}\PY{p}{[}\PY{l+s+s2}{\PYZdq{}}\PY{l+s+s2}{mode}\PY{l+s+s2}{\PYZdq{}}\PY{p}{,}\PY{l+s+s2}{\PYZdq{}}\PY{l+s+s2}{mean}\PY{l+s+s2}{\PYZdq{}}\PY{p}{]}\PY{p}{)}
\end{Verbatim}


\begin{Verbatim}[commandchars=\\\{\}]
{\color{outcolor}Out[{\color{outcolor}11}]:} <matplotlib.legend.Legend at 0x1a1cf35fd0>
\end{Verbatim}
            
    \begin{center}
    \adjustimage{max size={0.9\linewidth}{0.9\paperheight}}{output_32_1.png}
    \end{center}
    { \hspace*{\fill} \\}
    
    \begin{Verbatim}[commandchars=\\\{\}]
{\color{incolor}In [{\color{incolor}12}]:} \PY{n}{arr} \PY{o}{=} \PY{p}{(}\PY{n}{df}\PY{p}{[}\PY{l+s+s2}{\PYZdq{}}\PY{l+s+s2}{train\PYZus{}rows}\PY{l+s+s2}{\PYZdq{}}\PY{p}{]} \PY{o}{==} \PY{l+m+mi}{10000}\PY{p}{)} \PY{o}{\PYZam{}} \PY{p}{(}\PY{n}{df}\PY{p}{[}\PY{l+s+s2}{\PYZdq{}}\PY{l+s+s2}{max\PYZus{}new\PYZus{}feat}\PY{l+s+s2}{\PYZdq{}}\PY{p}{]} \PY{o}{==} \PY{l+m+mi}{500}\PY{p}{)} \PY{o}{\PYZam{}} \PY{p}{(}\PY{n}{df}\PY{p}{[}\PY{l+s+s2}{\PYZdq{}}\PY{l+s+s2}{transactionRevenue}\PY{l+s+s2}{\PYZdq{}}\PY{p}{]} \PY{o}{==} \PY{l+m+mi}{0}\PY{p}{)}
         \PY{n}{l} \PY{o}{=} \PY{n}{df}\PY{o}{.}\PY{n}{loc}\PY{p}{[}\PY{n}{arr}\PY{p}{,}\PY{l+s+s2}{\PYZdq{}}\PY{l+s+s2}{grouping\PYZus{}mode\PYZus{}cats}\PY{l+s+s2}{\PYZdq{}}\PY{p}{]}
         
         
         \PY{n}{l} \PY{o}{=} \PY{n}{df}\PY{o}{.}\PY{n}{loc}\PY{p}{[}\PY{n}{arr}\PY{p}{,}\PY{l+s+s2}{\PYZdq{}}\PY{l+s+s2}{final\PYZus{}score}\PY{l+s+s2}{\PYZdq{}}\PY{p}{]}
         \PY{n}{m} \PY{o}{=} \PY{n+nb}{min}\PY{p}{(}\PY{n}{l}\PY{p}{)}
         \PY{n}{m}
         
         \PY{n}{df}\PY{o}{.}\PY{n}{loc}\PY{p}{[}\PY{n}{df}\PY{p}{[}\PY{l+s+s2}{\PYZdq{}}\PY{l+s+s2}{final\PYZus{}score}\PY{l+s+s2}{\PYZdq{}}\PY{p}{]} \PY{o}{==} \PY{n}{m}\PY{p}{,}\PY{l+s+s2}{\PYZdq{}}\PY{l+s+s2}{grouping\PYZus{}mode\PYZus{}cats}\PY{l+s+s2}{\PYZdq{}}\PY{p}{]}
\end{Verbatim}


\begin{Verbatim}[commandchars=\\\{\}]
{\color{outcolor}Out[{\color{outcolor}12}]:} 13    mode
         14    mode
         Name: grouping\_mode\_cats, dtype: object
\end{Verbatim}
            
    \subsubsection{Conclusioni}\label{conclusioni}

La moda è stata testata solo su piccoli dataset(max 10000 righe) per il
suo alto costo computazionale(7min su 10000righe, 34min su 100000righe
senza contare il costo del caricare i dati da had, inoltre ha
complessità alta poichè con l'aumentare delle righe aumenta anche la
variabilità di ogni singola colonna e con il OHE si aggiungono colonne a
destra). Detto ciò per piccoli datasets dà molto scarto al funzionale
statistico della media in quanto a precisione.

    \subsection{Esperimento 6}\label{esperimento-6}

\begin{center}\rule{0.5\linewidth}{\linethickness}\end{center}

\subsubsection{Aumentare le foglie di LGBM o LGBM\_rf non porta
miglioramento dopo un certo
treshold}\label{aumentare-le-foglie-di-lgbm-o-lgbm_rf-non-porta-miglioramento-dopo-un-certo-treshold}

Si è notato che nonostante aumentare le foglie ad un albero di
regressione teoricamente dovrebbe dargli più potere espressivo in realtà
ad un certo punto non apporta pi alcun miglioramento poichè
probabilmente viene richiesto un certo supporto per creare una foglia e
l' albero nonlo crea oppure lo crea ma molti dati comunque cadono nelle
foglie più votate.

    \begin{Verbatim}[commandchars=\\\{\}]
{\color{incolor}In [{\color{incolor}13}]:} \PY{k+kn}{import} \PY{n+nn}{matplotlib}\PY{n+nn}{.}\PY{n+nn}{pyplot} \PY{k}{as} \PY{n+nn}{plt}
         \PY{k+kn}{from} \PY{n+nn}{generic} \PY{k}{import} \PY{n}{norm}
         
         \PY{n}{fig}\PY{p}{,} \PY{n}{ax} \PY{o}{=} \PY{n}{plt}\PY{o}{.}\PY{n}{subplots}\PY{p}{(} \PY{n}{figsize}\PY{o}{=}\PY{p}{(}\PY{l+m+mi}{9}\PY{p}{,}\PY{l+m+mi}{3}\PY{p}{)}\PY{p}{,} \PY{n}{tight\PYZus{}layout}\PY{o}{=}\PY{k+kc}{True}\PY{p}{)}
         
         \PY{n}{arr} \PY{o}{=} \PY{p}{(}\PY{n}{df}\PY{p}{[}\PY{l+s+s2}{\PYZdq{}}\PY{l+s+s2}{type}\PY{l+s+s2}{\PYZdq{}}\PY{p}{]} \PY{o}{==} \PY{l+s+s2}{\PYZdq{}}\PY{l+s+s2}{LightGBM}\PY{l+s+s2}{\PYZdq{}}\PY{p}{)} \PY{o}{\PYZam{}} \PY{p}{(}\PY{n}{df}\PY{p}{[}\PY{l+s+s2}{\PYZdq{}}\PY{l+s+s2}{train\PYZus{}rows}\PY{l+s+s2}{\PYZdq{}}\PY{p}{]} \PY{o}{==} \PY{l+m+mi}{10000}\PY{p}{)} \PY{o}{\PYZam{}} \PY{p}{(}\PY{n}{df}\PY{p}{[}\PY{l+s+s2}{\PYZdq{}}\PY{l+s+s2}{max\PYZus{}new\PYZus{}feat}\PY{l+s+s2}{\PYZdq{}}\PY{p}{]} \PY{o}{==} \PY{l+m+mi}{500}\PY{p}{)} \PY{o}{\PYZam{}} \PY{p}{(}\PY{n}{df}\PY{p}{[}\PY{l+s+s2}{\PYZdq{}}\PY{l+s+s2}{transactionRevenue}\PY{l+s+s2}{\PYZdq{}}\PY{p}{]} \PY{o}{==} \PY{l+m+mi}{0}\PY{p}{)} \PY{o}{\PYZam{}} \PY{p}{(}\PY{n}{df}\PY{p}{[}\PY{l+s+s2}{\PYZdq{}}\PY{l+s+s2}{grouping\PYZus{}mode\PYZus{}cats}\PY{l+s+s2}{\PYZdq{}}\PY{p}{]} \PY{o}{==} \PY{l+s+s2}{\PYZdq{}}\PY{l+s+s2}{mean}\PY{l+s+s2}{\PYZdq{}}\PY{p}{)} \PY{o}{\PYZam{}} \PY{p}{(}\PY{n}{df}\PY{p}{[}\PY{l+s+s2}{\PYZdq{}}\PY{l+s+s2}{feature\PYZus{}fraction}\PY{l+s+s2}{\PYZdq{}}\PY{p}{]} \PY{o}{==} \PY{l+m+mf}{0.99}\PY{p}{)} \PY{o}{\PYZam{}} \PY{p}{(}\PY{n}{df}\PY{p}{[}\PY{l+s+s2}{\PYZdq{}}\PY{l+s+s2}{grouping\PYZus{}mode\PYZus{}cats}\PY{l+s+s2}{\PYZdq{}}\PY{p}{]} \PY{o}{==} \PY{l+s+s2}{\PYZdq{}}\PY{l+s+s2}{mean}\PY{l+s+s2}{\PYZdq{}}\PY{p}{)} \PY{o}{\PYZam{}} \PY{p}{(}\PY{n}{df}\PY{p}{[}\PY{l+s+s2}{\PYZdq{}}\PY{l+s+s2}{bagging\PYZus{}fraction}\PY{l+s+s2}{\PYZdq{}}\PY{p}{]} \PY{o}{==} \PY{l+m+mf}{0.99}\PY{p}{)}
         \PY{n}{l} \PY{o}{=} \PY{n}{df}\PY{o}{.}\PY{n}{loc}\PY{p}{[}\PY{n}{arr}\PY{p}{,}\PY{l+s+s2}{\PYZdq{}}\PY{l+s+s2}{n\PYZus{}leaves}\PY{l+s+s2}{\PYZdq{}}\PY{p}{]}
         \PY{n}{ax}\PY{o}{.}\PY{n}{plot}\PY{p}{(} \PY{n}{norm}\PY{p}{(}\PY{n}{l}\PY{p}{)} \PY{p}{,}\PY{l+s+s1}{\PYZsq{}}\PY{l+s+s1}{r\PYZhy{}*}\PY{l+s+s1}{\PYZsq{}}\PY{p}{)}
         \PY{c+c1}{\PYZsh{}foglie in rosso}
         \PY{c+c1}{\PYZsh{}al diminuire delle foglie il punteggio cala cioè peggiora}
         
         \PY{n}{l} \PY{o}{=} \PY{n}{df}\PY{o}{.}\PY{n}{loc}\PY{p}{[}\PY{n}{arr}\PY{p}{,}\PY{l+s+s2}{\PYZdq{}}\PY{l+s+s2}{final\PYZus{}score}\PY{l+s+s2}{\PYZdq{}}\PY{p}{]}
         \PY{n}{ax}\PY{o}{.}\PY{n}{plot}\PY{p}{(} \PY{l+m+mi}{1}\PY{o}{/}\PY{n}{norm}\PY{p}{(}\PY{n}{l}\PY{p}{)} \PY{p}{)}
         
         \PY{n}{ax}\PY{o}{.}\PY{n}{legend}\PY{p}{(}\PY{p}{[}\PY{l+s+s2}{\PYZdq{}}\PY{l+s+s2}{n\PYZus{}leaves}\PY{l+s+s2}{\PYZdq{}}\PY{p}{,}\PY{l+s+s2}{\PYZdq{}}\PY{l+s+s2}{score}\PY{l+s+s2}{\PYZdq{}}\PY{p}{]}\PY{p}{)}
\end{Verbatim}


\begin{Verbatim}[commandchars=\\\{\}]
{\color{outcolor}Out[{\color{outcolor}13}]:} <matplotlib.legend.Legend at 0x1a1cf9fe10>
\end{Verbatim}
            
    \begin{center}
    \adjustimage{max size={0.9\linewidth}{0.9\paperheight}}{output_36_1.png}
    \end{center}
    { \hspace*{\fill} \\}
    
    Vedendo il grafico non è facile capire che per un aumento delle foglie
non consegue un miglioramento dello score perciò facciamo un
interrogazione al dataframe volta ad estrarre il valore minimo dello
score e il conseguente numero di foglie che implica quell' ottimo
punteggio, noteremo che ci sono più valori diversi del parametro
n\_leaves, questo perchè appunto aumentare le foglie non è detto porti a
un incremento dello score sopra un certo treshold

    \begin{Verbatim}[commandchars=\\\{\}]
{\color{incolor}In [{\color{incolor}14}]:} \PY{n}{arr} \PY{o}{=} \PY{p}{(}\PY{n}{df}\PY{p}{[}\PY{l+s+s2}{\PYZdq{}}\PY{l+s+s2}{type}\PY{l+s+s2}{\PYZdq{}}\PY{p}{]} \PY{o}{==} \PY{l+s+s2}{\PYZdq{}}\PY{l+s+s2}{LightGBM}\PY{l+s+s2}{\PYZdq{}}\PY{p}{)} \PY{o}{\PYZam{}} \PY{p}{(}\PY{n}{df}\PY{p}{[}\PY{l+s+s2}{\PYZdq{}}\PY{l+s+s2}{train\PYZus{}rows}\PY{l+s+s2}{\PYZdq{}}\PY{p}{]} \PY{o}{==} \PY{l+m+mi}{10000}\PY{p}{)} \PY{o}{\PYZam{}} \PY{p}{(}\PY{n}{df}\PY{p}{[}\PY{l+s+s2}{\PYZdq{}}\PY{l+s+s2}{max\PYZus{}new\PYZus{}feat}\PY{l+s+s2}{\PYZdq{}}\PY{p}{]} \PY{o}{==} \PY{l+m+mi}{500}\PY{p}{)} \PY{o}{\PYZam{}} \PY{p}{(}\PY{n}{df}\PY{p}{[}\PY{l+s+s2}{\PYZdq{}}\PY{l+s+s2}{transactionRevenue}\PY{l+s+s2}{\PYZdq{}}\PY{p}{]} \PY{o}{==} \PY{l+m+mi}{0}\PY{p}{)} \PY{o}{\PYZam{}} \PY{p}{(}\PY{n}{df}\PY{p}{[}\PY{l+s+s2}{\PYZdq{}}\PY{l+s+s2}{grouping\PYZus{}mode\PYZus{}cats}\PY{l+s+s2}{\PYZdq{}}\PY{p}{]} \PY{o}{==} \PY{l+s+s2}{\PYZdq{}}\PY{l+s+s2}{mean}\PY{l+s+s2}{\PYZdq{}}\PY{p}{)} \PY{o}{\PYZam{}} \PY{p}{(}\PY{n}{df}\PY{p}{[}\PY{l+s+s2}{\PYZdq{}}\PY{l+s+s2}{feature\PYZus{}fraction}\PY{l+s+s2}{\PYZdq{}}\PY{p}{]} \PY{o}{==} \PY{l+m+mf}{0.99}\PY{p}{)} \PY{o}{\PYZam{}} \PY{p}{(}\PY{n}{df}\PY{p}{[}\PY{l+s+s2}{\PYZdq{}}\PY{l+s+s2}{grouping\PYZus{}mode\PYZus{}cats}\PY{l+s+s2}{\PYZdq{}}\PY{p}{]} \PY{o}{==} \PY{l+s+s2}{\PYZdq{}}\PY{l+s+s2}{mean}\PY{l+s+s2}{\PYZdq{}}\PY{p}{)} \PY{o}{\PYZam{}} \PY{p}{(}\PY{n}{df}\PY{p}{[}\PY{l+s+s2}{\PYZdq{}}\PY{l+s+s2}{bagging\PYZus{}fraction}\PY{l+s+s2}{\PYZdq{}}\PY{p}{]} \PY{o}{==} \PY{l+m+mf}{0.99}\PY{p}{)}
         \PY{n}{l} \PY{o}{=} \PY{n}{df}\PY{o}{.}\PY{n}{loc}\PY{p}{[}\PY{n}{arr}\PY{p}{,}\PY{l+s+s2}{\PYZdq{}}\PY{l+s+s2}{n\PYZus{}leaves}\PY{l+s+s2}{\PYZdq{}}\PY{p}{]}
         
         \PY{n}{l} \PY{o}{=} \PY{n}{df}\PY{o}{.}\PY{n}{loc}\PY{p}{[}\PY{n}{arr}\PY{p}{,}\PY{l+s+s2}{\PYZdq{}}\PY{l+s+s2}{final\PYZus{}score}\PY{l+s+s2}{\PYZdq{}}\PY{p}{]}
         \PY{n}{m} \PY{o}{=} \PY{n+nb}{min}\PY{p}{(}\PY{n}{l}\PY{p}{)}
         \PY{n}{m}
         
         \PY{n}{df}\PY{o}{.}\PY{n}{loc}\PY{p}{[}\PY{n}{df}\PY{p}{[}\PY{l+s+s2}{\PYZdq{}}\PY{l+s+s2}{final\PYZus{}score}\PY{l+s+s2}{\PYZdq{}}\PY{p}{]} \PY{o}{==} \PY{n}{m}\PY{p}{,}\PY{l+s+s2}{\PYZdq{}}\PY{l+s+s2}{n\PYZus{}leaves}\PY{l+s+s2}{\PYZdq{}}\PY{p}{]}
\end{Verbatim}


\begin{Verbatim}[commandchars=\\\{\}]
{\color{outcolor}Out[{\color{outcolor}14}]:} 108    512
         109     16
         110     32
         111     64
         112    128
         113    256
         114    512
         Name: n\_leaves, dtype: int64
\end{Verbatim}
            
    \subsubsection{Conclusioni}\label{conclusioni}

I dati supportano il fatto che oltre un certo treshold non serve a nulla
aumentare le foglie, il corollario di questo fatto è che possiamo
ottenere un ottimo punteggio con poche foglie perciò risparmiando
computazione e spazio


    % Add a bibliography block to the postdoc
    
    
    
    \end{document}
